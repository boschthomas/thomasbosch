% This is LLNCS.DEM the demonstration file of
% the LaTeX macro package from Springer-Verlag
% for Lecture Notes in Computer Science,
% version 2.4 for LaTeX2e as of 16. April 2010
%
\documentclass{llncs}

% allows for temporary adjustment of side margins
\usepackage{chngpage}

% just makes the table prettier (see \toprule, \bottomrule, etc. commands below)
\usepackage{booktabs}

\usepackage[utf8]{inputenc}

% URL handling
\usepackage{url}
\urlstyle{same}

% Todos
%\usepackage[colorinlistoftodos]{todonotes}
%\newcommand{\ke}[1]{\todo[size=\small, color=orange!40]{\textbf{Kai:} #1}}
%\newcommand{\tb}[1]{\todo[size=\small, color=green!40]{\textbf{Thomas:} #1}}


%\usepackage{makeidx}  % allows for indexgeneration

%\usepackage{amsmath}
\usepackage{amsmath, amssymb}
\usepackage{mathabx}

% monospace within text
\newcommand{\ms}[1]{\texttt{#1}}

% examples
\usepackage{fancyvrb}
\DefineVerbatimEnvironment{ex}{Verbatim}{numbers=left,numbersep=2mm,frame=single,fontsize=\scriptsize}

\newenvironment{gcotable-1}{
  \scriptsize
  \sffamily
  \vspace{0.3cm}\linebreak
  \begin{tabular}{l|l|l|l|l|l}
  \hline
  \textbf{c. type} & \textbf{context concept} & \textbf{p. list} & \textbf{concepts} & \textbf{c. element} & \textbf{c. value} \\
  \hline

}{
  \hline
  \end{tabular}
  \linebreak
}

\newenvironment{gcotable}{
  \scriptsize
  \sffamily
  \vspace{0.3cm}\linebreak
  \begin{tabular}{l|l|l|l|l|l|l}
  \hline
  \textbf{c. type} & \textbf{context concept} & \textbf{left p. list} & \textbf{right p. list} & \textbf{concepts} & \textbf{c. element} & \textbf{c. value} \\
  \hline

}{
  \hline
  \end{tabular}
  \linebreak
}

\newenvironment{DL}{
  \scriptsize
  \sffamily
  \vspace{0.3cm}\linebreak
  \begin{tabular}{l}
	\textbf{DL:} \\

}{
  \end{tabular}
  \linebreak
}

\usepackage{xspace}
% Einfache und doppelte Anfuehrungszeichen
\newcommand{\qs}{``} 
\newcommand{\qe}{''\xspace} 
\newcommand{\sqs}{`} 
\newcommand{\sqe}{'\xspace} 

% checkmark
\usepackage{tikz}
\def\checkmark{\tikz\fill[scale=0.4](0,.35) -- (.25,0) -- (1,.7) -- (.25,.15) -- cycle;} 

% Xs
\usepackage{pifont}

% Tabellenabstände kleiner
\setlength{\intextsep}{10pt} % Vertical space above & below [h] floats
\setlength{\textfloatsep}{10pt} % Vertical space below (above) [t] ([b]) floats
% \setlength{\abovecaptionskip}{0pt}
% \setlength{\belowcaptionskip}{0pt}

\usepackage{tabularx}
\newcommand{\hr}{\hline\noalign{\smallskip}} % für die horizontalen linien in tabellen

% pipe
%\usepackage[T1]{fontenc}

% Todos
\usepackage[colorinlistoftodos]{todonotes}
\newcommand{\ke}[1]{\todo[size=\small, color=orange!40]{\textbf{Kai:} #1}}
\newcommand{\tb}[1]{\todo[size=\small, color=green!40]{\textbf{Thomas:} #1}}

\setcounter{secnumdepth}{5}

\begin{document}

%
%
\title{XXXXX}
%
\titlerunning{XXXXX}  % abbreviated title (for running head)
%                                     also used for the TOC unless
%                                     \toctitle is used
%
\author{XXXXX\inst{1} \and XXXXX\inst{2}}
%
\authorrunning{XXXXX} % abbreviated author list (for running head)
%
%%%% list of authors for the TOC (use if author list has to be modified)
\institute{XXXXX\\
\email{XXXXX},\\ 
\and
XXXXX \\
\email{XXXXX} 
}

\maketitle              % typeset the title of the contribution

\begin{abstract}


\keywords{..}
\end{abstract}
%

% ---------------

\section{Ideas}

\begin{itemize}
	\item check quality of thesauri using validation
\end{itemize}

\textbf{validations:}
\begin{itemize}
	\item how far the thesaurus conforms to established ISO norms for thesauri
	\item hierarchical relations between terms and relations
	\item classification of terms
	\item check links to other thesauri: STW Thesaurus for Economics of the ZBW, AGROVOC thesaurus of FAO, DBpedia
	\item typical knowledge modeling patterns occurring by the use of SKOS and SKOS-XL
	\item ambiguous terms: equivalence and compound equivalence relationships between preferred and non-preferred terms
	\item representation of compound relationships and compound concept/ compound concepts are part of the ISO2788 standard
\end{itemize}

\textbf{validation of datasets conversion to SKOS:}
\begin{itemize}
	\item are terms and relations mapped adequately to SKOS class and properties? 
	\item completeness
	\item relations between generic and ambiguous terms (having different meanings in specialized contexts) 
\end{itemize}

\textbf{TheSoz specific validations:}
\begin{itemize}
	\item TheSoz extensions / own namespace / TheSoz specific classes and properties
\end{itemize}

\section{Validations}

\textbf{map term to multiple terms not possible in SKOS:}
A concept in one thesaurus
might correspond to a combination of two concepts in
another thesaurus, e.g. the term "Electronic Government" of the TheSoz has originally been mapped to
the combination of the terms "Public Administration"
and "Internet" of the STW. The mapping properties of
SKOS do not allow such single-to-multiple relations
(neither for one language nor for multiple languages).

\section{Motivation}

\section{Visualization Ideas}

ich erstelle ja constraint violation triples, die so aussehen:

\begin{ex}
CONSTRUCT {
    _:violation
        a spin:ConstraintViolation ;
        rdfs:label ?violationMessage ;
        spin:violationRoot ?violationRoot ;
        spin:violationPath ?violationPath ;
        spin:violationSource ?violationSource ;
        spin:fix ?violationFix ;
        :validationLevel :error }
\end{ex}

Auch den validation level möchte ich angeben, da wird es sicher was geben, zumindest irgendwann.
Ich nutzte in der zwischenzeit spin:validationLevel obwohl es das im spin namespace (noch) nicht gibt, würde aber sehr viel soinn machen.

Wir haben dann als ergebnis der validierung einen graphen mit constraint validation tripeln.
diese kann man nach validation level (e.g. error, fail, message, ...) sortieren, unterteilen und auch getrennt visualisieren.

use case:
visualisierung nur der wirklich wirklich wichtigen (fail) constraint violation triples

use case:
visualisierung einer bestimmten constraint violation triple gruppe

-----
da ich immer auch den root, also das subject von constraint violations mit generiere, kann man auch eine visualisierung erstellen bestimmter rdf subjects, die der grund für constraint violations darstellen, wieder sortiert nach validation level

----
soertierung nach violation source ist auch denkbar, also welche constraint ausgedrückt durch welche DSCL hat welche (wieviele) constraint violations verursacht (spin:violationSource).

----

spin:fix zeigt auf triple die angeben, wie eine constraint violation behoben werden kann, auch das kann visualisiert werden.
z.b. eine kette von aktionen, die dazu führen, dass bestimmte constraint violations behoben werden können.

--

das sprudeln die ideen...

-----

spin:path --> welche properties sind an einem constraint violation beteiligt. hier könnte man einen pfad aufzeigen, der für eine bestimmte constraint violation zeigt, welche properties in welcher reihenflge beteiligt sind. eine property chain

--

 also mit all diesen generierten informationen lässt sich viel anfangen und sinnvolle dinge visualisieren.
wir können auch mal drüber sprechen.

---
macht es sinn auch die constraints selber zu visualisieren?

- welche constraints gibt es die für eine bestimmte DSCL in SPIN definiert wurden?
- wenn eine specific constraint aug eine generic constraint gemappt wird, können wir visualisieren wie spezifische constraints zueinander in beziehung stehen und auch zu anderen spezifischen constraints anderer DSCLs
- macht eine visualisierung der SPARQL queries sinn die verwendet werden constraints zu validieren?
- welche specific und generic constraints sind welchen requirements zugeordnet, kömnte auch visualisiert werden...

ich denke das reicht jetzt erst mal....

\bibliography{../../literature/literature}{}
\bibliographystyle{plain}
\setcounter{tocdepth}{1}
%\listoftodos
\end{document}
