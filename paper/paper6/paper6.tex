%% JWS example using Elsevier elsart classes, adapted by Tim 
%% Finin, finin@cs.umbc.edu

%% comment out one of these to get preprint vs. journal format

%\documentclass{elsart}       %% one column for review, preprint
\documentclass{elsart3p}    %% two columns for publication

\usepackage{graphicx,amssymb}

\renewcommand\floatpagefraction{.2}
\makeatletter
\def\elsartstyle{%
    \def\normalsize{\@setfontsize\normalsize\@xiipt{14.5}}
    \def\small{\@setfontsize\small\@xipt{13.6}}
    \let\footnotesize=\small
    \def\large{\@setfontsize\large\@xivpt{18}}
    \def\Large{\@setfontsize\Large\@xviipt{22}}
    \skip\@mpfootins = 18\p@ \@plus 2\p@
    \normalsize
}
\@ifundefined{square}{}{\let\Box\square}
\makeatother

%% tell Latex where to look for files menioned in \includegraphics
\graphicspath{{jwsGraphics/}}

% Todos
\usepackage[colorinlistoftodos]{todonotes}
\newcommand{\ke}[1]{\todo[size=\small, color=orange!40]{\textbf{Kai:} #1}}
\newcommand{\tb}[1]{\todo[size=\small, color=green!40]{\textbf{Thomas:} #1}}
\newcommand{\bz}[1]{\todo[size=\small, color=red!40]{\textbf{Ben:} #1}}
\newcommand{\jw}[1]{\todo[size=\small, color=blue!40]{\textbf{Achim:} #1}}

% examples
\usepackage{fancyvrb}
\DefineVerbatimEnvironment{ex}{Verbatim}{numbers=left,numbersep=2mm,frame=single,fontsize=\scriptsize}

\newenvironment{DL}{
  %\scriptsize
  %\sffamily
  \vspace{0cm}
  \begin{tabular}{l l}

}{
  \end{tabular}
  %\linebreak
}

\begin{document}

\begin{frontmatter}

%% This TITLE has a note reference (titleNote) whose 
%% contants is specified about 10 lines later.

\title{Modeling and Validation of Metadata on \\ Highly-Complex Person-Level Data Sets in RDF}

%% Here are the three AUTHORS.  Note the separating commas
%% with no and.

\author[gesis]{Thomas Bosch\corauthref{C}},
\ead{thomas.bosch@gesis.org}
\author[gesis]{Benjamin Zapilko},
\ead{benjamin.zapilko@gesis.org}
\author[gesis]{Joachim Wackerow}
\ead{joachim.wackerow@gesis.org},
\author[mannheim]{Kai Eckert}
\ead{kai@informatik.uni-mannheim.de}

%% these specify any 'notes' for the title and authors,
%% including 'thanks' that acknowledge support, etc. and an
%% indication of who the corresponding author is if there's
%% more than one author.

%\thanks[titleNote]{Adapted from a document written by Simon Pepping}
%\thanks[fininGrants]{Partisl support provided by NSF award NSF-ITR-IIS-0326460} 
\corauth[C]{Corresponding author. Tel: + 49 (0) 621 / 1246-271}

\medskip

%% Here are the addresses refereenced for the authors.  If
%% all of the authors are from the same institution you need
%% not have the lables for the author and address commands.

\address[gesis]{Gesis - Leibniz Institute for the Social Sciences, 68159 Mannheim, Germany}
\address[mannheim]{University of Mannheim, 68159 Mannheim, Germany}

\begin{abstract} 



\end{abstract}

%% you choose your own keywords

\begin{keyword}
 RDF Validation, Person-Level Metadata, DDI-RDF Discovery Vocabulary, Disco, Linked Data
\end{keyword}

\end{frontmatter}


\section{Motivation}

\begin{itemize}
  \item why do we need these newly defined vocabularies?
	\item Why do we need RDF Validation?
	\item Why do we need to validate metadata on person-level data sets? 
\end{itemize}

\textbf{Contributions.}
\begin{itemize}
  \item We show how to model highly complex person-level data sets in RDF using different vocabularies 
  \item We define constraints on Disco, Data Cube, PHDD, DCAT, XKOS data sets
	\item WE provide an implementation to validate Disco, Data Cube, PHDD, DCAT, XKOS data sets 
\end{itemize}

\section{Data Documentation Initiative}

\section{DDI-RDF Discovery Vocabulary}

\begin{itemize}
  \item overview of disco
	\item description of disco
\end{itemize}

\subsection{Conceptional Model}

\subsection{Use Cases}

\subsection{Relationships to other Vocabularies}

\begin{itemize}
	\item RDF Data Cube, DCAT, PHDD, XKOS
	\item brief description of vocabularies
\end{itemize}

\section{Validation Environment}

\section{Validation of Metadata on Person-Level Data Sets}

\textcolor{blue}{multiple validation levels}

Where appropriate constraint types are related to complementary requirements of the RDF validation requirements database.

We state one Disco constraint for each constraint type.

Constraint types marked with an asterisk can be used as an OWL 2 axiom.
Thus, reasoners may be used to infer implicit triples resolving constraint violations causing when these constraints are validated.
\textcolor{blue}{example}

\subsection{Data Model Consistency}

Is the data consistent with the intended semantics of the data model?
Ensures the integrity of the data according to the data model.

\subsection{Unqualified Cardinality Restrictions}

\subsection{Minimum Qualified Cardinality Restrictions}

A minimum cardinality restrictions contains all those individuals that are connected by a property to at least n different individuals/literals 
that are instances of a particular class or data range. If the class is missing, it is taken to be owl:Thing. 
If the data range is missing, it is taken to be rdfs:Literal.
$\geq n R. C$ is a minimum qualified cardinality restriction where $n \in \mathbb{N}$\footnote{{\em R-75-MINIMUM-QUALIFIED-CARDINALITY-ON-PROPERTIES} and {\em R-211-CARDINALITY-CONSTRAINTS}}.
\textbf{{\em DISCO-C-XX:}}
A {\em disco:Questionnaire} has at least 1 {\em disco:question} relationship to {\em disco:Question}\footnote{When mapped to DL, we do not state namespace prefixes for simplicity reasons}.

\begin{DL}
Questionnaire $\sqsubseteq$ $\geq$1 question.Question
\end{DL}

\subsection{Exact Qualified Cardinality Restrictions}

An exact cardinality restriction contains all those individuals that are connected by a property to exactly n different individuals that are instances of a particular class or data range. 
If the class is missing, it is taken to be owl:Thing. 
If the data range is not present, it is taken to be rdfs:Literal.
$\geq n R. C \sqcap \leq n R. C $ is an exact qualified cardinality restriction where $n \in \mathbb{N}$\footnote{{\em R-74-EXACT-QUALIFIED-CARDINALITY-ON-PROPERTIES} and {\em R-211-CARDINALITY-CONSTRAINTS}.}.
\textbf{{\em DISCO-C-XX:}}
A {\em disco:Question} has exactly 1 {\em disco:universe} relationship to {\em disco:Universe}.

\begin{DL}
Question $\sqsubseteq$ \\
$\geq$1 universe.Universe $\sqcap$ $\leq$1 universe.Universe \\
\end{DL}

\subsection{Universal Quantifications*}

A universal class expression ({\em value restriction} in DL) contains all those individuals that are connected by an object property only to individuals that are instances of a particular class\footnote{{\em R-91-UNIVERSAL-QUANTIFICATION-ON-PROPERTIES}}.
\textbf{{\em DISCO-C-XX:}}
Only {\em disco:LogicalDataSet}s can have {\em disco:aggregation} relationships to {\em qb:DataSet}s.

\begin{DL}
LogicalDataSet $\sqsubseteq$ $\forall$ aggregation.DataSet \\
\end{DL}

\subsection{Existential Quantifications}

An existential class expression ({\em existential restriction} in DL terminology) contains all those individuals that are connected by the property P to an individual x that is an instance of the class C or to literals that are in the data range DR\footnote{{\em R-86-EXISTENTIAL-QUANTIFICATION-ON-PROPERTIES}}.
\textbf{{\em DISCO-C-XX:}} 
There must be at least 1 {\em disco:universe} relationship from {\em disco:Studies} or {\em disco:StudyGroups} to {\em disco:Universe}.

\begin{DL}
Study $\sqcup$ StudyGroup $\sqsubseteq$ $\exists$ universe.Universe \\
\end{DL}

\subsection{Disjunction}

A union class expression contains all individuals that are instances of at least one class $C_{i}$ for 1 $\leq$ i $\leq$ n. 
A union data range contains all tuples of literals that are contained in the at least one data range $DR_{i}$ for 1 $\leq$ i $\leq$ n.
Synonyms of {\em disjunction} are {\em union} and {\em inclusive or}\footnote{{\em R-17-DISJUNCTION-OF-CLASS-EXPRESSIONS} and {\em R-18-DISJUNCTION-OF-DATA-RANGES}}.
\textbf{{\em DISCO-C-XX:}} 
Only {\em disco:Variable}s or {\em disco:Question}s or {\em disco:RepresentedVariable}s can have {\em disco:concept} relationships to {\em skos:Concept}s.

\begin{DL}
Variable $\sqcup$ Question $\sqcup$ RepresentedVariable \\
$\sqsubseteq$ $\forall$ concept.Concept \\
\end{DL}
		
\subsection{Property Domains*}

{\em Property Domains} ({\em domain restrictions on roles} in DL) restricts the domain of object and data properties.
The purpose is to declare that a given property is associated with a class. 
In OO terms this is the declaration of a member, field, attribute or association. 
$\exists R. \top \sqsubseteq C$ is the object property restriction where $R$ is the object property (role) whose domain is restricted to concept $C$\footnote{{\em R-25-OBJECT-PROPERTY-DOMAIN} and {\em R-26-DATA-PROPERTY-DOMAIN}}.
{\em Property Domain} constraints are defined for each Disco object and data property.
\textbf{{\em DISCO-C-XX:}} 
Only {\em disco:Question}s can have {\em disco:responseDomain} relationships.

\begin{DL}
$\exists$ responseDomain.$\top$ $\sqsubseteq$ Question 
\end{DL}

\subsection{Property Ranges*}

{\em Property Range} ({\em range restrictions on roles} in DL) restricts the range of object and data properties.
$\top \sqsubseteq \forall R . C$ is the range restriction to the object property $R$ (restricted by the concept $C$)\footnote{{\em R-28-OBJECT-PROPERTY-RANGE} and {\em R-35-DATA-PROPERTY-RANGE}}. 
\textbf{{\em DISCO-C-XX:}} 
{\em disco:caseQuantity} relationships can only point to literals of the datatype {\em xsd:nonNegativeInteger}.

\begin{DL}
$\top$ $\sqsubseteq$ $\forall$ caseQuantity.nonNegativeInteger \\
\end{DL}

\subsection{Class-Specific Property Ranges}		

{\em Class-Specific Property Range} restricts the range of object and data properties for individuals within a specific context (e.g. class, shape, application profile).
The values of each member property of a class may be limited by their value type, such as xsd:string or foaf:Person\footnote{{\em R-29-CLASS-SPECIFIC-RANGE-OF-RDF-OBJECTS} and {\em R-36-CLASS-SPECIFIC-RANGE-OF-RDF-LITERALS}}. 
\textbf{{\em DISCO-C-XX:}} 
Only {\em disco:Question}s can have {\em disco:questionText} relationships to literals of the datatype {\em rdf:langString}.

\begin{DL}
$\neg$Question $\sqsubseteq$ $\neg\exists$ questionText.langString
\end{DL}

\subsection{Subsumption*}

A subclass axiom ({\em concept inclusion} in DL) states that the class C1 is a subclass of the class C2 - C1 is more specific than C2\footnote{{\em R-100-SUBSUMPTION}}.
\textbf{{\em DISCO-C-XX:}} 
All {\em disco:Universe}s must also be {\em skos:Concept}s.

\begin{DL}
Universe $\sqsubseteq$ Concept
\end{DL}

\subsection{Class Equivalence*}

{\em Class Equivalence} asserts that two concepts have the same instances.
While synonyms are an obvious example of equivalent concepts, in practice one more
often uses concept equivalence to give a name to complex expressions \cite{Kroetzsch2012}.
Concept equivalence is indeed subsumption from left and right ($A \sqsubseteq B$ and $B \sqsubseteq A$ implies $A \equiv B$)\footnote{{\em R-3-EQUIVALENT-CLASSES}}.
\textbf{{\em DISCO-C-XX:}}
All {\em sio:SIO\_000367}s must also be {\em disco:Variable}s.

\begin{DL}
Variable $\equiv$ SIO\_000367
\end{DL}

The Semanticscience Integrated Ontology (SIO)\footnote{https://code.google.com/p/semanticscience/wiki/SIO} provides a simple, integrated ontology of types and relations for rich description of objects, processes and their attributes.
{\em sio:SIO\_000367} is a variable defined as a value that may change within the scope of a given problem or set of operations.
Thus, {\em sio:SIO\_000367} is equivalent to {\em disco:Variable}.

\subsection{Sub Properties*}

{\em Sub Properties} state that the property P1 is a sub property of the property P2 - that is, if an individual x is connected by P1 to an individual or a literal y, then x is also connected by P2 to y. 
\textbf{{\em DISCO-C-XX:}}
If an individual x is connected by {\em disco:fundedBy} to an individual y, then x is also connected by {\em dcterms:contributor} to y. 

\begin{DL}
fundedBy $\sqsubseteq$ contributor 
\end{DL}

\subsection{Disjoint Classes}

{\em Disjoint Classes} state that all of the classes are pairwise disjoint; 
that is, no individual can be at the same time an instance of these disjoint classes\footnote{{\em R-7-DISJOINT-CLASSES}}.
\textbf{{\em DISCO-C-XX:}} 
All Disco classes are defined to be pairwise disjoint.
The following DL statements holds for each pair of Disco classes:

\begin{DL}
Study $\sqcap$ Variable $\sqsubseteq$ $\perp$\\
\end{DL}

\subsection{Asymmetric Object Properties}

A property is asymmetric if it is disjoint from its own inverse \cite{Kroetzsch2012}.
An object property asymmetry axiom states that the object property OP is asymmetric - that is, if an individual x is connected by OP to an individual y, then y cannot be connected by OP to x\footnote{{\em R-62-ASYMMETRIC-OBJECT-PROPERTIES}}. 
\textbf{{\em DISCO-C-XX:}} 
A {\em disco:Variable} may be based on a {\em disco:RepresentedVariable}.
A {\em disco:RepresentedVariable}, however, cannot be based on a {\em disco:Variable}.
This is a kind of mistake may occur as a semantically equivalent object property for the other direction may also be possible ({\em disco:basisOf}).

\begin{DL}
$basedOn \sqcap basedOn^{-} \sqsubseteq \bot$ 
\end{DL}

\subsection{Class-Specific Irreflexive Object Properties}

A property is irreflexive if it is never locally reflexive \cite{Kroetzsch2012}.
An object property irreflexivity axiom states that the object property OP is irreflexive - that is, no individual is connected by OP to itself. 
\textbf{{\em DISCO-C-XX:}}
Within the Disco context, {\em skos:Concept}s cannot be related via the object property {\em skos:boader} to themselves (DL: Concept $\sqsubseteq$ $\neg$$\exists$ broader.Self).

\textbf{{\em DISCO-C-XX:}}
Within the Disco context, {\em skos:Concept}s cannot be related via the object property {\em skos:narrower} to themselves. 

\begin{DL}
Concept $\sqsubseteq$ $\neg$$\exists$ narrower.Self. 
\end{DL}

\subsection{Context-Specific Exclusive OR of Property Groups}

Exclusive or is a logical operation that outputs true whenever both inputs differ (one is true, the other is false).
Only one of multiple property groups leads to valid data\footnote{{\em R-13-DISJOINT-GROUP-OF-PROPERTIES-CLASS-SPECIFIC}}.
\textbf{{\em DISCO-C-XX:}}
Within the context of Disco, skos:Concepts can have either skos:definition (when interpreted as DDI concepts) or skos:notation and skos:prefLabel properties (when interpreted as DDI codes and categories), but not both.

\begin{DL}
Concept $\sqsubseteq$ ($\neg$ D $\sqcap$ C) $\sqcup$ (D $\sqcap$ $\neg$ C) \\ 
D $\equiv$ A $\sqcap$ B \\
A $\sqsubseteq$ $\geq$ 1 notation.string $\sqcap$ $\leq$ 1 notation.string \\
B $\sqsubseteq$ $\geq$ 1 prefLabel.string $\sqcap$ $\leq$ 1 prefLabel.string \\
C $\sqsubseteq$ $\geq$ 1 definition.string $\sqcap$ $\leq$ 1 definition.string \\
\end{DL}

\subsection{Allowed Values}

It is a common requirement to narrow down the value space of a property by an exhaustive enumeration of the valid values (both literals or resource). 
This is often rendered in drop down boxes or radio buttons in user interfaces. 
Allowed values for properties can be IRIs, IRIs (matching one or multiple patterns), (any) literals, literals of a list of allowed literals (e.g. 'red' 'blue' 'green'), typed literals one or multiple type(s) (e.g. xsd:string)\footnote{{\em R-30-ALLOWED-VALUES-FOR-RDF-OBJECTS} and 
{\em R-37-ALLOWED-VALUES-FOR-RDF-LITERALS}}.

\textbf{{\em DISCO-C-XX}}.
{\em disco:CategoryStatistics} can only have {\em disco:computationBase} relationships to the values 'valid' and 'invalid' of the datatype {\em rdf:langString}.

\begin{DL}
CategoryStatistics $\equiv$ \\ $\forall$ computationBase.\{valid,invalid\} $\sqcap$ langString \\
\end{DL}

\subsection{Membership in Controlled Vocabularies.}

Resources can only be members of listed controlled vocabularies\footnote{{\em R-32-MEMBERSHIP-OF-RDF-OBJECTS-IN-CONTROLLED-VOCABULARIES} and
{\em R-39-MEMBERSHIP-OF-RDF-LITERALS-IN-CONTROLLED-VOCABULARIES}}.

\textbf{{\em DISCO-C-XX}}.
{\em disco:SummaryStatistics} can only have {\em disco:summaryStatisticType} relationships to {\em skos:Concept}s which must be members of the controlled vocabulary {\em ddicv:SummaryStatisticType} which is a {\em skos:ConceptScheme}.

\begin{DL}
SummaryStatistics $\sqsubseteq$ \\
$\forall summaryStatisticType.Concept \sqcap \forall inScheme . A$ \\
$A \equiv ConceptScheme \sqcap \{SummaryStatisticType\}$
\end{DL}

\subsection{Literal Value Comparison}

Depending on the property semantics,
there are cases where two different literal values must have
a specific ordering with respect to an operator. 
P1 and P2 are the datatype properties we need to compare and 
OP is the comparison operator (\textless, \textless=, \textgreater, \textgreater=, =, !=)\footnote{{\em R-43-LITERAL-VALUE-COMPARISON}}.
The {\em COMP Pattern}, one of the Data Quality Test Pattern, can be used to validate the {\em Literal Value Comparison} constraint \cite{Kontokostas2014}:

\begin{ex}
SELECT ?s WHERE { 
    ?s %%P1%% ?v1 .
    ?s %%P2%% ?v2 .
    FILTER ( ?v1 %%OP%% ?v2 ) }
\end{ex}

\textbf{{\em DISCO-C-XX}}.
{\em disco:startDate}s must be before (‘\textless’) the {\em disco:endDate}s.
To validate this constraint we bind the variables as follows (P1: {\em disco:startDate}, P2: {\em disco:endDate}, OP: \textless). 

\subsection{Literal Ranges}

P1 is a data property (of an instance of class C1) and its literal value must be between the range of [$V_{min}$,$V_{max}$].
\textbf{{\em DISCO-C-XX:}}
{\em disco:percentage} (domain: {\em disco:CategoryStatistics}) literals must be of the datatype {\em xsd:double} whose range should be restricted to be between 0 and 100 (not expressible in DL)\footnote{{\em R-45-RANGES-OF-RDF-LITERAL-VALUES}}.

\subsection{Mathematical Operations}

Examples for {\em Mathematical Operations} are the addition of two dates, the addition of days to a start date, and statistical computations (e.g. average, mean, sum).
\textbf{{\em DISCO-C-XX:}}
The sum of {\em disco:percentage} {\em xsd:double} values of all codes (represented as {\em skos:Concept}s) of a code list ({\em skos:ConceptScheme} or {\em skos:OrderedCollection}), serving as representation of a particular {\em disco:Variable}, must exactly be 100  
(not expressible in DL)\footnote{{\em R-42-MATHEMATICAL-OPERATIONS} and {\em R-41-STATISTICAL-COMPUTATIONS}}.

\subsection{Default Values}

Default values for objects and literals are inferred automatically.
It should be possible to declare the default value for a given property, e.g. so that input forms can be pre-populated and to insert a required property that is missing in a web service call\footnote{{\em R-31-DEFAULT-VALUES-OF-RDF-OBJECTS} and {\em R-38-DEFAULT-VALUES-OF-RDF-LITERALS}}.
\textbf{{\em DISCO-C-XX:}}
The value 'true' for the property {\em disco:isPublic} ({\em xsd:boolean}) indicates that the data set ({\em disco:LogicalDataSet}) can be accessed (usually downloaded) by anyone.
Per default, access to data sets should be restricted ('false').

\subsection{Language Tag Matching}

For particular data properties, values must be stated for predefined languages (not expressible in DL)\footnote{{\em R-47-LANGUAGE-TAG-MATCHING}}.
\textbf{{\em DISCO-C-XX:}}
There must be an English variable name ({\em skos:notation}) for each {\em disco:Variable} within {\em disco:LogicalDataSet}s.

\subsection{Language Tag Cardinality}

For particular data properties, values of predefined languages must be stated for determined number of times (not expressible in DL)
\footnote{{\em R-49-RDF-LITERALS-HAVING-AT-MOST-ONE-LANGUAGE-TAG} and {\em R-48-MISSING-LANGUAGE-TAGS}}.
\textbf{{\em DISCO-C-XX:}}
There must be at least one English {\em disco:questionText} for each {\em disco:Question} within {\em disco:LogicalDataSet}s.
\textbf{{\em DISCO-C-XX:}}
There should be at most one English literal value for variable names ({\em skos:notation}, domain: {\em disco:Variable}).

\subsection{Conditional Properties}

If specific properties exist, then specific other properties also have to be present\footnote{{\em R-71-CONDITIONAL-PROPERTIES}}.
\textbf{{\em DISCO-C-XX:}}
If a {\em skos:Concept represents a code (having a {\em skos:notation} property) and a category (having a {\em skos:prefLabel} property), 
then the property {\em disco:isValid} has to be stated indicating if the code is valid ('true') or missing ('false').

\subsection{Recommended Properties}

Which properties are not required but recommended within a particular context\footnote{{\em R-72-RECOMMENDED-PROPERTIES}}.
\textbf{{\em DISCO-C-XX:}}
The property ({\em skos:notation} is not mandatory for {\em disco:Variable}s, but recommended to indicate variable names.

\subsection{Value is Valid for Datatype}

Make sure that a value is valid for its datatype.
It has to be ensured, e.g., that a date is really a date, or that a xsd:nonNegativeInteger value is not negative. 
\textbf{{\em DISCO-C-XX:}}
Check if literal values of the property {\em disco:startDate} are really of the datatype {\em xsd:date}(not expressible in DL).

\subsection{Equivalent Properties}

\subsection{Functional Properties}

\subsection{Data Property Facets}

xsd:length of disco:purpose

\subsection{Hierarchies}

Hierarchies of DDI concepts

\begin{itemize}
	\item root?
\end{itemize}

\subsection{Ordering}

In DDI, variables, questions, and categories are typically organized in a particular order. 
For obtaining this order, {\em skos:OrderedCollection} resources are used. 
A collection of variables, e.g, is represented as being of the type {\em skos:OrderedCollection} containing multiple variables (each represented as {\em skos:Concept}) in a {\em skos:memberList}. 

\section{Validation in Combination with other Vocabularies}

\subsection{RDF Data Cube Vocabulary}

There are 22 RDF Data Cube integrity constraints\footnote{http://www.w3.org/TR/vocab-data-cube/\#wf} \cite{CyganiakReynolds2014}.

\begin{itemize}
	\item \textcolor{blue}{describe 1 constraint of each type in detail with Datalog query}
\end{itemize}

\textbf{Data Model Consistency.}
{\em IC-0}, \textcolor{red}{{\em IC-8}}, {\em IC-13}, {\em IC-14}, {\em IC-15}, {\em IC-16}, {\em IC-17}, \textcolor{red}{{\em IC-18}}
%\begin{itemize}
	%\item {\em IC-0:} Datatypes must be consistent under RDF D-entailment using a datatype map containing all the datatypes used within the graph.
	%\item \textcolor{red}{{\em IC-8:} Every {\em qb:componentProperty} on a {\em qb:SliceKey} must also be declared as a {\em qb:component} of the associated {\em qb:DataStructureDefinition}.} 
%\end{itemize}

\textbf{Unqualified Cardinality Restrictions.}
{\em IC-4}, {\em IC-5}, {\em IC-10}, {\em IC-11}
%\begin{itemize}
	%\item {\em IC-4:} Every dimension declared in a {\em qb:DataStructureDefinition} must have a declared {\em rdfs:range}. 
	%\item {\em IC-5:} Every dimension with range {\em skos:Concept} must have a {\em qb:codeList}. 
	%\item {\em IC-10:} Every {\em qb:Slice} must have a value for every dimension declared in its {\em qb:sliceStructure}. 
	%\item {\em IC-11:} All dimensions required - Every {\em qb:Observation} has a value for each dimension declared in its associated {\em qb:DataStructureDefinition}.
%\end{itemize}

\textbf{Qualified Cardinality Restrictions.}
{\em IC-1}, {\em IC-2}, {\em IC-3}, {\em IC-6}, {\em IC-7}, {\em IC-9} 
%\begin{itemize}
	%\item {\em IC-1:} Every {\em qb:Observation} has exactly one associated {\em qb:DataSet}. 
	%\item {\em IC-2:} Every {\em qb:DataSet} has exactly one associated {\em qb:DataStructureDefinition}. 
	%\item {\em IC-3:} Every {\em qb:DataStructureDefinition} must include at least one declared measure. 
	%\item {\em IC-6:} The only components of a {\em qb:DataStructureDefinition} that may be marked as optional, using {\em qb:componentRequired} are attributes. 
	%\item {\em IC-7:} Every {\em qb:SliceKey} must be associated with a {\em qb:DataStructureDefinition}. 
	%\item {\em IC-9:} Each {\em qb:Slice} must have exactly one associated {\em qb:sliceStructure}. 
%\end{itemize}

\textbf{Duplicate Detection.}
{\em IC-12}
%\begin{itemize}
	%\item {\em IC-12:} No duplicate observations - No two {\em qb:Observations} in the same {\em qb:DataSet} may have the same value for all dimensions. 
%\end{itemize}

\textbf{Membership in Controlled Vocabularies.}
\textcolor{red}{{\em IC-19}}

\textbf{SKOS Hierarchies.}
{\em IC-20}, {\em IC-21}

%\textbf{Mandatory Properties.}

%\textbf{Optional Properties.}

\begin{itemize}
	\item \textcolor{blue}{are there further constraints which can be described in this paper?}
\end{itemize}

\subsection{SKOS and XKOS}

\subsection{PHDD}

\subsection{DCAT}

\section{DDI 4 Validation}

\section{Implementation}

\textbf{Validation of Data Cube instances.}
\begin{itemize}
	\item \textcolor{blue}{is there an implementation available yet? If not I implement it in our validation environment}
	\item \textcolor{blue}{example QB data set / we could use the one from the QB spec}
\end{itemize}

\section{Evaluation}

\begin{itemize}
	\item validate constraints on complex example data set
	\item count Disco, QB, PHDD, XKOS, DCAT constraints by constraint groups 
\end{itemize}

\section{Related Work}

\textbf{RDF Data Cube Vocabulary.}
A well-formed RDF Data Cube is an a RDF graph describing one or more instances of {\em qb:DataSet} for which each of the defined integrity constraints passes.
Each integrity constraint is expressed as narrative prose and, where possible, a SPARQL ASK query or query template. 
If the ASK query is applied to an RDF graph then it will return true if that graph contains one or more Data Cube instances which violate the corresponding constraint.
\cite{CyganiakReynolds2014}

\textbf{DCAT.}
\textcolor{blue}{are there already constraints defined for DCAT?}

\section{Conclusion and Future Work}

\bibliographystyle{elsart-num-sort}

\bibliography{../../literature/literature}

\end{document}
