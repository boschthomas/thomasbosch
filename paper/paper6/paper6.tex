%% JWS example using Elsevier elsart classes, adapted by Tim 
%% Finin, finin@cs.umbc.edu

%% comment out one of these to get preprint vs. journal format

%\documentclass{elsart}       %% one column for review, preprint
\documentclass{elsart3p}    %% two columns for publication

\usepackage{graphicx,amssymb}

\renewcommand\floatpagefraction{.2}
\makeatletter
\def\elsartstyle{%
    \def\normalsize{\@setfontsize\normalsize\@xiipt{14.5}}
    \def\small{\@setfontsize\small\@xipt{13.6}}
    \let\footnotesize=\small
    \def\large{\@setfontsize\large\@xivpt{18}}
    \def\Large{\@setfontsize\Large\@xviipt{22}}
    \skip\@mpfootins = 18\p@ \@plus 2\p@
    \normalsize
}
\@ifundefined{square}{}{\let\Box\square}
\makeatother

%% tell Latex where to look for files menioned in \includegraphics
\graphicspath{{jwsGraphics/}}

% Todos
\usepackage[colorinlistoftodos]{todonotes}
\newcommand{\ke}[1]{\todo[size=\small, color=orange!40]{\textbf{Kai:} #1}}
\newcommand{\tb}[1]{\todo[size=\small, color=green!40]{\textbf{Thomas:} #1}}
\newcommand{\bz}[1]{\todo[size=\small, color=red!40]{\textbf{Ben:} #1}}
\newcommand{\jw}[1]{\todo[size=\small, color=blue!40]{\textbf{Achim:} #1}}

\newenvironment{DL}{
  %\scriptsize
  %\sffamily
  \vspace{0cm}
  \begin{tabular}{l l}

}{
  \end{tabular}
  %\linebreak
}

\begin{document}

\begin{frontmatter}

%% This TITLE has a note reference (titleNote) whose 
%% contants is specified about 10 lines later.

\title{RDF Validation of \\ Metadata on Highly-Complex Person-Level Data Sets}

%% Here are the three AUTHORS.  Note the separating commas
%% with no and.

\author[gesis]{Thomas Bosch\corauthref{C}},
\ead{thomas.bosch@gesis.org}
\author[gesis]{Benjamin Zapilko},
\ead{benjamin.zapilko@gesis.org}
\author[gesis]{Joachim Wackerow}
\ead{joachim.wackerow@gesis.org},
\author[mannheim]{Kai Eckert}
\ead{kai@informatik.uni-mannheim.de}

%% these specify any 'notes' for the title and authors,
%% including 'thanks' that acknowledge support, etc. and an
%% indication of who the corresponding author is if there's
%% more than one author.

%\thanks[titleNote]{Adapted from a document written by Simon Pepping}
%\thanks[fininGrants]{Partisl support provided by NSF award NSF-ITR-IIS-0326460} 
\corauth[C]{Corresponding author. Tel: + 49 (0) 621 / 1246-271}

\medskip

%% Here are the addresses refereenced for the authors.  If
%% all of the authors are from the same institution you need
%% not have the lables for the author and address commands.

\address[gesis]{Gesis - Leibniz Institute for the Social Sciences, 68159 Mannheim, Germany}
\address[mannheim]{University of Mannheim, 68159 Mannheim, Germany}

\begin{abstract} 



\end{abstract}

%% you choose your own keywords

\begin{keyword}
 RDF Validation, Person-Level Metadata, DDI-RDF Discovery Vocabulary, Disco, Linked Data
\end{keyword}

\end{frontmatter}


\section{Motivation}

\begin{itemize}
	\item Why do we need RDF Validation?
	\item Why do we need to validate metadata on person-level data sets? 
\end{itemize}

\textbf{Contributions.}
\begin{itemize}
  \item We define constraints on Disco, Data Cube, PHDD, DCAT, XKOS data sets
	\item WE provide an implementation to validate Disco, Data Cube, PHDD, DCAT, XKOS data sets 
\end{itemize}

\section{Data Documentation Initiative}

\section{DDI-RDF Discovery Vocabulary}

\begin{itemize}
  \item overview of disco
	\item description of disco
\end{itemize}

\subsection{Conceptional Model}

\subsection{Use Cases}

\subsection{Relationships to other Vocabularies}

\begin{itemize}
	\item RDF Data Cube, DCAT, PHDD, XKOS
	\item brief description of vocabularies
\end{itemize}

\section{Validation Environment}

\section{Validation of Metadata on Person-Level Data Sets}

\subsection{Data Model Consistency}

Is the data consistent with the intended semantics of the data model?
Ensures the integrity of the data according to the data model.

\subsection{Unqualified Cardinality Restrictions}

\subsection{Minimum Qualified Cardinality Restrictions}

{\em DISCO-C-XX}. 
A {\em disco:Questionnaire} has at least 1 {\em disco:question} relationship to {\em disco:Question}.

\begin{DL}
Questionnaire $\sqsubseteq$ $\geq$1 question.Question
\end{DL}

\subsection{Exact Qualified Cardinality Restrictions}

{\em DISCO-C-XX}. 
A {\em disco:Question} has exactly 1 {\em disco:universe} relationship to {\em disco:Universe}.

\begin{DL}
Question $\sqsubseteq$ \\
$\geq$1 universe.Universe $\sqcap$ $\leq$1 universe.Universe \\
\end{DL}

\subsection{Universal Quantifications}

{\em DISCO-C-XX}. 
Only {\em disco:LogicalDataSet}s can have {\em disco:aggregation} relationships to {\em qb:DataSet}s.

\begin{DL}
LogicalDataSet $\sqsubseteq$ $\forall$ aggregation.DataSet \\
\end{DL}

\subsection{Existential Quantifications}

{\em DISCO-C-XX}. 
There must be at least 1 {\em disco:universe} relationship from {\em disco:Studies} or {\em disco:StudyGroups} to {\em disco:Universe}.

\begin{DL}
Study $\sqcup$ StudyGroup $\sqsubseteq$ $\exists$ universe.Universe \\
\end{DL}

\subsection{SKOS-Related Constraints}

\subsubsection{Membership in Controlled Vocabularies.}

Questions, variables, and represented variables may have representations.

\subsubsection{Hierarchies}

Hierarchies of DDI concepts

\begin{itemize}
	\item root?
\end{itemize}

\subsection{Ordering}

In DDI, variables, questions, and categories are typically organized in a particular order. 
For obtaining this order, {\em skos:OrderedCollection} resources are used. 
A collection of variables, e.g, is represented as being of the type {\em skos:OrderedCollection} containing multiple variables (each represented as {\em skos:Concept}) in a {\em skos:memberList}. 

\section{Validation in Combination with other Vocabularies}

\subsection{RDF Data Cube Vocabulary}

There are 22 RDF Data Cube integrity constraints\footnote{http://www.w3.org/TR/vocab-data-cube/\#wf} \cite{CyganiakReynolds2014}.

\begin{itemize}
	\item \textcolor{blue}{describe 1 constraint of each type in detail with Datalog query}
\end{itemize}

\textbf{Data Model Consistency.}
{\em IC-0}, \textcolor{red}{{\em IC-8}}, {\em IC-13}, {\em IC-14}, {\em IC-15}, {\em IC-16}, {\em IC-17}, \textcolor{red}{{\em IC-18}}
%\begin{itemize}
	%\item {\em IC-0:} Datatypes must be consistent under RDF D-entailment using a datatype map containing all the datatypes used within the graph.
	%\item \textcolor{red}{{\em IC-8:} Every {\em qb:componentProperty} on a {\em qb:SliceKey} must also be declared as a {\em qb:component} of the associated {\em qb:DataStructureDefinition}.} 
%\end{itemize}

\textbf{Unqualified Cardinality Restrictions.}
{\em IC-4}, {\em IC-5}, {\em IC-10}, {\em IC-11}
%\begin{itemize}
	%\item {\em IC-4:} Every dimension declared in a {\em qb:DataStructureDefinition} must have a declared {\em rdfs:range}. 
	%\item {\em IC-5:} Every dimension with range {\em skos:Concept} must have a {\em qb:codeList}. 
	%\item {\em IC-10:} Every {\em qb:Slice} must have a value for every dimension declared in its {\em qb:sliceStructure}. 
	%\item {\em IC-11:} All dimensions required - Every {\em qb:Observation} has a value for each dimension declared in its associated {\em qb:DataStructureDefinition}.
%\end{itemize}

\textbf{Qualified Cardinality Restrictions.}
{\em IC-1}, {\em IC-2}, {\em IC-3}, {\em IC-6}, {\em IC-7}, {\em IC-9} 
%\begin{itemize}
	%\item {\em IC-1:} Every {\em qb:Observation} has exactly one associated {\em qb:DataSet}. 
	%\item {\em IC-2:} Every {\em qb:DataSet} has exactly one associated {\em qb:DataStructureDefinition}. 
	%\item {\em IC-3:} Every {\em qb:DataStructureDefinition} must include at least one declared measure. 
	%\item {\em IC-6:} The only components of a {\em qb:DataStructureDefinition} that may be marked as optional, using {\em qb:componentRequired} are attributes. 
	%\item {\em IC-7:} Every {\em qb:SliceKey} must be associated with a {\em qb:DataStructureDefinition}. 
	%\item {\em IC-9:} Each {\em qb:Slice} must have exactly one associated {\em qb:sliceStructure}. 
%\end{itemize}

\textbf{Duplicate Detection.}
{\em IC-12}
%\begin{itemize}
	%\item {\em IC-12:} No duplicate observations - No two {\em qb:Observations} in the same {\em qb:DataSet} may have the same value for all dimensions. 
%\end{itemize}

\textbf{Membership in Controlled Vocabularies.}
\textcolor{red}{{\em IC-19}}

\textbf{SKOS Hierarchies.}
{\em IC-20}, {\em IC-21}

%\textbf{Mandatory Properties.}

%\textbf{Optional Properties.}

\begin{itemize}
	\item \textcolor{blue}{are there further constraints which can be described in this paper?}
\end{itemize}

\subsection{SKOS and XKOS}

\subsection{PHDD}

\subsection{DCAT}

\section{DDI 4 Validation}

\section{Implementation}

\textbf{Validation of Data Cube instances.}
\begin{itemize}
	\item \textcolor{blue}{is there an implementation available yet? If not I implement it in our validation environment}
	\item \textcolor{blue}{example QB data set / we could use the one from the QB spec}
\end{itemize}

\section{Evaluation}

\begin{itemize}
	\item validate constraints on complex example data set
\end{itemize}

\section{Related Work}

\textbf{RDF Data Cube Vocabulary.}
A well-formed RDF Data Cube is an a RDF graph describing one or more instances of {\em qb:DataSet} for which each of the defined integrity constraints passes.
Each integrity constraint is expressed as narrative prose and, where possible, a SPARQL ASK query or query template. 
If the ASK query is applied to an RDF graph then it will return true if that graph contains one or more Data Cube instances which violate the corresponding constraint.
\cite{CyganiakReynolds2014}

\textbf{DCAT.}
\textcolor{blue}{are there already constraints defined for DCAT?}

\section{Conclusion and Future Work}

\bibliographystyle{elsart-num-sort}

\bibliography{../../literature/literature}

\end{document}
