%%%%%%%%%%%%%%%%%%%%%%%%%%%%%%%%%%%%%%%%%%%%%%%%%%%%%%%%%%%%%%%%%%%%%%%%%%%%
%% Trim Size: 9.75in x 6.5in
%% Text Area: 8in (include Runningheads) x 5in
%% ws-ijsc.tex     14-2-07
%% Tex file to use with ws-ijsc.cls written in Latex2E. 
%% The content, structure, format and layout of this style file is the 
%% property of World Scientific Publishing Co. Pte. Ltd. 
%% Copyright 1995, 2002 by World Scientific Publishing Co. 
%% All rights are reserved.
%%%%%%%%%%%%%%%%%%%%%%%%%%%%%%%%%%%%%%%%%%%%%%%%%%%%%%%%%%%%%%%%%%%%%%%%%%%%
%%

\documentclass{ws-ijsc}

\begin{document}

\markboth{Authors' Names}
{Instructions for Typing Manuscripts (Paper's Title)}

%%%%%%%%%%%%%%%%%%%%% Publisher's Area please ignore %%%%%%%%%%%%%%%
%
\catchline{}{}{}{}{}
%
%%%%%%%%%%%%%%%%%%%%%%%%%%%%%%%%%%%%%%%%%%%%%%%%%%%%%%%%%%%%%%%%%%%%

\title{Summary of the new added content as well as the differences between the journal version and the conference version}



\maketitle



\begin{abstract}

\end{abstract}

\section{Introduction}

\begin{itemlist}
 \item Minor changes
\end{itemlist}

\section{Common Vocabularies in the SBE Sciences}	

\begin{itemlist}
 \item We give an in-depth description of common vocabularies in the SBE sciences and not just a too short overview on them.
 \item We describe a complete running example how SBE sciences data and metadata is represented in RDF.
 \item We introduce a representative RDF validation case study.
 \item We inserted a paragraph about Data Catalog Vocabulary (DCAT) which enables to represent data sets inside of data collections like portals, repositories, catalogs, and archives which serve as typical entry points when searching for data.
\end{itemlist}

\section{Related Work}

\begin{itemlist}
 \item No changes
\end{itemlist}

\section{Classification of Constraint Types and Constraints}	

\begin{itemlist}
 \item Complete walk through and minor corrections
\end{itemlist}

\section{Implementation}

\begin{itemlist}
 \item We meticulously described how we implemented our validation environment which can directly be used to validate RDF data against constraints expressed in any RDF-based constraint language and extracted from or defined for any RDF vocabulary. 
\end{itemlist}

\section{Evaluation}

\begin{itemlist}
 \item Complete walk through and minor corrections
\end{itemlist}

\section{Conclusion and Future Work}

\begin{itemlist}
 \item One additional main finding
 \item 2 additional paragraphs
\end{itemlist}

\section{references}

\begin{itemlist}
 \item We included 20 new references of high-quality.
\end{itemlist}

\end{document}
