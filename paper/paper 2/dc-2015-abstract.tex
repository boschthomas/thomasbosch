% This is LLNCS.DEM the demonstration file of
% the LaTeX macro package from Springer-Verlag
% for Lecture Notes in Computer Science,
% version 2.4 for LaTeX2e as of 16. April 2010
%
\documentclass[a4paper,fontsize=11pt]{scrartcl}



% Anpassungen an DC Template
\usepackage[a4paper, left=2.9cm,right=2.87cm,bottom=3cm,top=3cm,headsep=1.27cm]{geometry}
\setlength{\parskip}{0.5em}
\renewcommand{\tablename}{TABLE} 
\renewcommand{\figurename}{FIG.} 
\usepackage{titlesec}
\titlespacing\section{0pt}{12pt plus 2pt minus 2pt}{6pt plus 1pt minus 1pt}
\titlespacing\subsection{0pt}{12pt plus 2pt minus 2pt}{3pt plus 1pt minus 1pt}
\addtokomafont{title}{\fontsize{14pt}{1em}\selectfont}
\setkomafont{section}{\fontsize{12pt}{0pt}\selectfont}
\addtokomafont{subsection}{\fontsize{11pt}{0pt}\selectfont}
% \addtokomafont{author}{\fontsize{12pt}{1em}\selectfont}
\usepackage{scrpage2}
\clearscrheadfoot
\pagestyle{scrheadings}
\renewcommand*{\titlepagestyle}{scrheadings}
\chead{\fontsize{9pt}{0pt}\selectfont Proc. Int'l Conf. on Dublin Core and Metadata Applications 2015}
\usepackage{apacite}


\date{}

\usepackage[utf8]{inputenc}

% URL handling
\usepackage{url}
\urlstyle{same}

% Todos
\usepackage[colorinlistoftodos]{todonotes}
\newcommand{\ke}[1]{\todo[size=\small, color=orange!40]{\textbf{Kai:} #1}}
\newcommand{\tb}[1]{\todo[size=\small, color=green!40]{\textbf{Thomas:} #1}}


% \usepackage{makeidx}  % allows for indexgeneration -- yes, but we don't need this

\usepackage{amsmath}

% monospace within text
\newcommand{\ms}[1]{\texttt{#1}}

% examples
\usepackage{fancyvrb}
\DefineVerbatimEnvironment{ex}{Verbatim}{numbers=left,numbersep=2mm,frame=single,fontsize=\scriptsize}

\usepackage{xspace}
% Einfache und doppelte Anfuehrungszeichen
\newcommand{\qs}{``} 
\newcommand{\qe}{''\xspace} 
\newcommand{\sqs}{`} 
\newcommand{\sqe}{'\xspace} 

% checkmark
\usepackage{tikz}
\def\checkmark{\tikz\fill[scale=0.4](0,.35) -- (.25,0) -- (1,.7) -- (.25,.15) -- cycle;} 

% Xs
\usepackage{pifont}

% Tabellenabstände kleiner
\setlength{\intextsep}{10pt} % Vertical space above & below [h] floats
\setlength{\textfloatsep}{10pt} % Vertical space below (above) [t] ([b]) floats
% \setlength{\abovecaptionskip}{0pt}
% \setlength{\belowcaptionskip}{0pt}

\usepackage{tabularx}
\newcommand{\hr}{\hline\noalign{\smallskip}} % für die horizontalen linien in tabellen

\newenvironment{DL}{
  %\scriptsize
  %\sffamily
  \vspace{0cm}
	\begin{center}
  \begin{tabular}{r l}

}{
  \end{tabular}
	\end{center}
}

% just makes the table prettier (see \toprule, \bottomrule, etc. commands below)
\usepackage{booktabs}

% Tabellenabstände kleiner
\setlength{\intextsep}{10pt} % Vertical space above & below [h] floats
\setlength{\textfloatsep}{10pt} % Vertical space below (above) [t] ([b]) floats
% \setlength{\abovecaptionskip}{0pt}
% \setlength{\belowcaptionskip}{0pt}

\usepackage{tabularx}

\usepackage{float}

\begin{document}

\title{\vspace{-1em}Guidance, please! Towards a framework for RDF-based constraint languages.}

\author{Thomas Bosch\\GESIS – Leibniz Institute \\for the Social Sciences, Germany\\thomas.bosch@gesis.org \and Kai Eckert\\Stuttgart Media University, Germany\\eckert@hdm-stuttgart.de}

\maketitle
\vspace{-3em}
\section*{Abstract}
In the context of the DCMI RDF Application Profile task group and the W3C Data Shapes Working Group solutions for the proper formulation of constraints and validation of RDF data on these constraints are developed. Several approaches and constraint languages exist but there is no clear favorite and none of the languages is able to meet all requirements raised by data practitioners.
To support the work, a comprehensive, community-driven database has been created where case studies, use cases, requirements and solutions are collected. Based on this database,
we published by today 81 types of constraints that are required by various stakeholders for data applications. We generally use this collection of constraint types to gain a better understanding of the expressiveness of existing solutions and gaps that still need to be filled. 
Regarding the implementation of constraint languages, we already proposed to use high-level languages to describe the constraints, but map them to SPARQL queries in order to execute the actual validation; we demonstrated this approach for Description Set Profiles.
In this paper, we generalize from the experience of implementing Description Set Profiles by introducing an abstraction layer that is able to describe any constraint type in a way that is more or less straight-forwardly transformable to SPARQL queries. 
It provides a basic terminology and classification system for RDF constraints to foster discussions on RDF validation.
We demonstrate that using another layer on top of SPARQL helps to implement validation consistently accross constraint languages and simplifies the actual implementation of new languages.


%\listoftodos
\end{document}
