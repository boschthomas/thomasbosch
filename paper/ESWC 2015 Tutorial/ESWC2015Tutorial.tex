% This is LLNCS.DEM the demonstration file of
% the LaTeX macro package from Springer-Verlag
% for Lecture Notes in Computer Science,
% version 2.4 for LaTeX2e as of 16. April 2010
%
\documentclass{llncs}

% allows for temporary adjustment of side margins
\usepackage{chngpage}

% just makes the table prettier (see \toprule, \bottomrule, etc. commands below)
\usepackage{booktabs}

\usepackage[utf8]{inputenc}
%\usepackage[font=small,skip=0pt]{caption}

% footnotes
\usepackage{scrextend}

% URL handling
\usepackage{url}
\urlstyle{same}

% Todos
%\usepackage[colorinlistoftodos]{todonotes}
%\newcommand{\ke}[1]{\todo[size=\small, color=orange!40]{\textbf{Kai:} #1}}
%\newcommand{\tb}[1]{\todo[size=\small, color=green!40]{\textbf{Thomas:} #1}}


%\usepackage{makeidx}  % allows for indexgeneration

%\usepackage{amsmath}
\usepackage{amsmath, amssymb}
\usepackage{mathabx}

% monospace within text
\newcommand{\ms}[1]{\texttt{#1}}

% examples
\usepackage{fancyvrb}
\DefineVerbatimEnvironment{ex}{Verbatim}{numbers=left,numbersep=2mm,frame=single,fontsize=\scriptsize}

\usepackage{xspace}
% Einfache und doppelte Anfuehrungszeichen
\newcommand{\qs}{``} 
\newcommand{\qe}{''\xspace} 
\newcommand{\sqs}{`} 
\newcommand{\sqe}{'\xspace} 

% checkmark
\usepackage{tikz}
\def\checkmark{\tikz\fill[scale=0.4](0,.35) -- (.25,0) -- (1,.7) -- (.25,.15) -- cycle;} 

% Xs
\usepackage{pifont}

% Tabellenabstände kleiner
\setlength{\intextsep}{10pt} % Vertical space above & below [h] floats
\setlength{\textfloatsep}{10pt} % Vertical space below (above) [t] ([b]) floats
% \setlength{\abovecaptionskip}{0pt}
% \setlength{\belowcaptionskip}{0pt}

\usepackage{tabularx}
\newcommand{\hr}{\hline\noalign{\smallskip}} % für die horizontalen linien in tabellen

% Todos
\usepackage[colorinlistoftodos]{todonotes}
\newcommand{\bz}[1]{\todo[size=\small, color=orange!40]{\textbf{Ben:} #1}}
\newcommand{\tb}[1]{\todo[size=\small, color=green!40]{\textbf{Thomas:} #1}}

\newenvironment{table-1cols}{
  \scriptsize
  \sffamily
  \vspace{0.3cm}
  \begin{tabular}{l}
  \hline
  \textbf{Requirements} \\
  \hline

}{
  \hline
  \end{tabular}
  \linebreak
}

\newenvironment{table-2cols}{
  \scriptsize
  \sffamily
  \vspace{0.3cm}
  \begin{tabular}{l|l}
  \hline
  \textbf{Requirements} & \textbf{Covering DSCLs} \\
  \hline

}{
  \hline
  \end{tabular}
  \linebreak
}

\newenvironment{complexity}{
  %\scriptsize
  %\sffamily
  %\vspace{0.3cm}
  \begin{tabular}{l|l}
  \hline
  \textbf{Complexity Class} & \textbf{Complexity} \\
  \hline

}{
  \hline
  \end{tabular}
  \linebreak
}

\newenvironment{DL}{
  %\scriptsize
  %\sffamily
  \vspace{0cm}
  \begin{tabular}{r l}

}{
  \end{tabular}
  %\linebreak
}


\newenvironment{evaluation}{
  %\scriptsize
  %\sffamily
  %\vspace{0.3cm}
  \begin{tabular}{l|c|c|c|c|c|c}
  \hline
  \textbf{Constraint Class} & \textbf{DSP} & \textbf{OWL2-DL} & \textbf{OWL2-QL} & \textbf{ReSh} & \textbf{ShEx} & \textbf{SPIN} \\
  \hline

}{
  \hline
  \end{tabular}
  \linebreak
}

\newenvironment{constraint-languages-complexity}{
  %\scriptsize
  %\sffamily
  %\vspace{0.3cm}
  \begin{tabular}{l|c|c|c|c|c|c}
  \hline
  \textbf{Complexity Class} & \textbf{DSP} & \textbf{OWL2-DL} & \textbf{OWL2-QL} & \textbf{ReSh} & \textbf{ShEx} & \textbf{SPIN} \\
  \hline

}{
  \hline
  \end{tabular}
  \linebreak
}

\newenvironment{user-fiendliness}{
  %\scriptsize
  %\sffamily
  %\vspace{0.3cm}
  \begin{tabular}{l|c|c|c|c|c}
  \hline
  \textbf{criterion} & \textbf{DSP} & \textbf{OWL2} & \textbf{ReSh} & \textbf{ShEx} & \textbf{SPIN} \\
  \hline

}{
  \hline
  \end{tabular}
  \linebreak
}

\setcounter{secnumdepth}{5}

\begin{document}

%
%
\title{Let’s Disco – Publishing person-level data as Linked Data.}
\subtitle{Building a Data Portal with DCAT, DDI-RDF Discovery, PHDD, and other vocabularies.\bz{Ben's Kommentar}
}
%
\titlerunning{XXXXX}  % abbreviated title (for running head)
%                                     also used for the TOC unless
%                                     \toctitle is used
%
\author{XXX\inst{1} \and XXX\inst{2}}
%
\authorrunning{XXXXX} % abbreviated author list (for running head)
%
%%%% list of authors for the TOC (use if author list has to be modified)
\institute{GESIS – Leibniz Institute for the Social Sciences, Germany\\
\email{thomas.bosch@gesis.org},\\ 
\and
University of Mannheim, Germany \\
\email{\{christian,erman,kai\}@informatik.uni-mannheim.de} 
\and
Albstadt-Sigmaringen University, Germany \\
\email{nolle@hs-albsig.de}
}

\maketitle              % typeset the title of the contribution

\begin{abstract}
200 words

\keywords{RDF Validation, RDF Constraints, OWL 2 QL, OWL 2 DL, Reasoning, RDF Validation Requirements, Linked Data, Semantic Web}
\end{abstract}
%

\section{Anweisungen}

Proposals should not exceed 5 pages using a reasonable font size (not less than 11 pts). They should include the call for participation together with the following details:
- Abstract: 200 words maximum, for inclusion to the ESWC 2015 Web site.
- Tutorial description: objectives of the tutorial and relevance to ESWC 2015; information about the scope and level of detail of the material to be covered; intended audiences; learning objectives; practical sessions.
- Tutorial length: Half or full day, including preferences.
- Previous versions or related tutorials: other venues of this or similar tutorials (potentially by a different team of tutors), motivation for offering the tutorial again, at the ESWC2015.
- Tutoring team: short bios of the presenters, including previous training and public speaking experience (e.g., teaching in English, conference presentations, tutorials etc.)

\section{Call for Participation}

The DDI-RDF Discovery Vocabulary (Disco) is a RDF Schema vocabulary that supports the discovery of microdata sets and related metadata using RDF technologies in the Web of Linked Data. Disco can be used to discover datasets by searching for specific questions, topics, and geographical coverage. Disco is intended to provide means to describe microdata by essential metadata for the discovery purpose.
 
The DDI (Data Documentation Initiative) is a structured metadata standard related to the observation and measurement of human activity.
It started out in the mid-1990s as a replacement for traditional archival code books documenting research data, and then branched off to cover the research data life cycle. Over time, as the data landscape has
changed, the DDI XML specifications have evolved to add new coverage and functionality to respond to new user requirements.
While the Data Documentation Initiative (DDI) is an international metadata standard with origins in the quantitative social sciences, it is increasingly being used by researchers and practitioners in other disciplines. The DDI specifications are also being used to document other data types, such as social media, biomarkers, administrative data, and transaction data. The specification itself is modular and can document and manage different stages of the data lifecycle, such as conceptualization, collection, processing, analysis, distribution, discovery, repurposing, and archiving.
Existing DDI-XML instances can be transformed into this RDF format and therefore exposed as Linked Data. This workshop aims to present the Disco vocabulary and its practical appliance with DDI metadata in detail.
 
For a publication of DDI metadata as Linked Data, widely accepted and adopted RDF vocabularies (e.g. RDF Data Cube, DCAT, and PHDD) are reused to a large extend. The workshop organizers show how Disco is interwoven with these vocabularies. The Data Cube vocabulary is a W3C standard for representing data cubes representing multidimensional aggregate data derived from microdata which is represented by Disco. DCAT is a W3C standard for describing catalogs of datasets. Physical data description (PHDD) represents data (tables) in a rectangular format. The data could be either represented in records with character-separated values (CSV) or in records with fixed length. The combined usage of PHDD, Disco, and DCAT enables the creation of data repositories which provide metadata for the description of collections, for data discovery, and for processing of the data.
 
In different real world use cases, the workshop organizers demonstrate the adoption and inclusion of Disco in existing information systems. Workshop participants are encouraged to present their own use cases where Disco has been applied or will be used. Together with the workshop organizers, workshop participants will have the possibility to elaborate an RDF representation of their use cases and to formulate typical queries which are necessary to solve use case related problems.
The workshop has a technical perspective and addresses software developers as well as data publishers who want to publish DDI metadata as Linked Data. No prerequisites are necessary, since a short introduction to RDF and Linked Open Data will be given.

\section{Tutorial Description}

\textbf{objectives of the tutorial} 

\textbf{relevance to ESWC 2015}

\textbf{information about the scope and level of detail of the material to be covered}

\textbf{intended audiences}

\textbf{learning objectives}

\textbf{practical sessions}

\section{Tutorial Length}

Half or full day, including preferences

\section{Previous Versions or Related Tutorials}

Thomas Bosch and Benjamin Zapilko hold a half-day tutorial at the EDDI14 – 6th Annual European DDI User Conference in London [1].
We got great feedback for this tutorial and had the idea to provide this tutorial at the ESWC conference next year as well.

\section{Tutoring Team}



\bibliography{../../literature/literature}{}
\bibliographystyle{plain}
\setcounter{tocdepth}{1}
%\listoftodos
\end{document}
