% This is LLNCS.DEM the demonstration file of
% the LaTeX macro package from Springer-Verlag
% for Lecture Notes in Computer Science,
% version 2.4 for LaTeX2e as of 16. April 2010
%
\documentclass{llncs}

% allows for temporary adjustment of side margins
\usepackage{chngpage}

% just makes the table prettier (see \toprule, \bottomrule, etc. commands below)
\usepackage{booktabs}

\usepackage[utf8]{inputenc}
%\usepackage[font=small,skip=0pt]{caption}

% footnotes
\usepackage{scrextend}

% URL handling
\usepackage{url}
\urlstyle{same}

% Todos
%\usepackage[colorinlistoftodos]{todonotes}
%\newcommand{\ke}[1]{\todo[size=\small, color=orange!40]{\textbf{Kai:} #1}}
%\newcommand{\tb}[1]{\todo[size=\small, color=green!40]{\textbf{Thomas:} #1}}


%\usepackage{makeidx}  % allows for indexgeneration

%\usepackage{amsmath}
\usepackage{amsmath, amssymb}
\usepackage{mathabx}

% monospace within text
\newcommand{\ms}[1]{\texttt{#1}}

% examples
\usepackage{fancyvrb}
\DefineVerbatimEnvironment{ex}{Verbatim}{numbers=left,numbersep=2mm,frame=single,fontsize=\scriptsize}

\usepackage{xspace}
% Einfache und doppelte Anfuehrungszeichen
\newcommand{\qs}{``} 
\newcommand{\qe}{''\xspace} 
\newcommand{\sqs}{`} 
\newcommand{\sqe}{'\xspace} 

% checkmark
\usepackage{tikz}
\def\checkmark{\tikz\fill[scale=0.4](0,.35) -- (.25,0) -- (1,.7) -- (.25,.15) -- cycle;} 

% Xs
\usepackage{pifont}

% Tabellenabstände kleiner
\setlength{\intextsep}{10pt} % Vertical space above & below [h] floats
\setlength{\textfloatsep}{10pt} % Vertical space below (above) [t] ([b]) floats
% \setlength{\abovecaptionskip}{0pt}
% \setlength{\belowcaptionskip}{0pt}

\usepackage{tabularx}
\newcommand{\hr}{\hline\noalign{\smallskip}} % für die horizontalen linien in tabellen

% Todos
\usepackage[colorinlistoftodos]{todonotes}
\newcommand{\bz}[1]{\todo[size=\small, color=orange!40]{\textbf{Ben:} #1}}
\newcommand{\tb}[1]{\todo[size=\small, color=green!40]{\textbf{Thomas:} #1}}

\newenvironment{table-1cols}{
  \scriptsize
  \sffamily
  \vspace{0.3cm}
  \begin{tabular}{l}
  \hline
  \textbf{Requirements} \\
  \hline

}{
  \hline
  \end{tabular}
  \linebreak
}

\newenvironment{table-2cols}{
  \scriptsize
  \sffamily
  \vspace{0.3cm}
  \begin{tabular}{l|l}
  \hline
  \textbf{Requirements} & \textbf{Covering DSCLs} \\
  \hline

}{
  \hline
  \end{tabular}
  \linebreak
}

\newenvironment{complexity}{
  %\scriptsize
  %\sffamily
  %\vspace{0.3cm}
  \begin{tabular}{l|l}
  \hline
  \textbf{Complexity Class} & \textbf{Complexity} \\
  \hline

}{
  \hline
  \end{tabular}
  \linebreak
}

\newenvironment{DL}{
  %\scriptsize
  %\sffamily
  \vspace{0cm}
  \begin{tabular}{r l}

}{
  \end{tabular}
  %\linebreak
}


\newenvironment{evaluation}{
  %\scriptsize
  %\sffamily
  %\vspace{0.3cm}
  \begin{tabular}{l|c|c|c|c|c|c}
  \hline
  \textbf{Constraint Class} & \textbf{DSP} & \textbf{OWL2-DL} & \textbf{OWL2-QL} & \textbf{ReSh} & \textbf{ShEx} & \textbf{SPIN} \\
  \hline

}{
  \hline
  \end{tabular}
  \linebreak
}

\newenvironment{constraint-languages-complexity}{
  %\scriptsize
  %\sffamily
  %\vspace{0.3cm}
  \begin{tabular}{l|c|c|c|c|c|c}
  \hline
  \textbf{Complexity Class} & \textbf{DSP} & \textbf{OWL2-DL} & \textbf{OWL2-QL} & \textbf{ReSh} & \textbf{ShEx} & \textbf{SPIN} \\
  \hline

}{
  \hline
  \end{tabular}
  \linebreak
}

\newenvironment{user-fiendliness}{
  %\scriptsize
  %\sffamily
  %\vspace{0.3cm}
  \begin{tabular}{l|c|c|c|c|c}
  \hline
  \textbf{criterion} & \textbf{DSP} & \textbf{OWL2} & \textbf{ReSh} & \textbf{ShEx} & \textbf{SPIN} \\
  \hline

}{
  \hline
  \end{tabular}
  \linebreak
}

\setcounter{secnumdepth}{5}

\begin{document}

%
%
\title{Let’s Disco – Publishing person-level data as Linked Data.}
\subtitle{Building a Data Portal with DCAT, DDI-RDF Discovery, PHDD, and other vocabularies.\bz{Ben's Kommentar}
}
%
\titlerunning{XXXXX}  % abbreviated title (for running head)
%                                     also used for the TOC unless
%                                     \toctitle is used
%
\author{XXX\inst{1} \and XXX\inst{1} \and Joachim Wackerow\inst{1}}
%
\authorrunning{XXXXX} % abbreviated author list (for running head)
%
%%%% list of authors for the TOC (use if author list has to be modified)
\institute{GESIS – Leibniz Institute for the Social Sciences, Germany\\
\email{\{thomas.bosch,benjamin.zapilko,joachim.wackerow\}}@gesis.org}

\maketitle              % typeset the title of the contribution

\begin{abstract}
The DDI-RDF Discovery Vocabulary (Disco) is an RDF Schema vocabulary that supports the discovery of microdata sets and related metadata using RDF technologies in the Web of Linked Data. Disco can be used to discover datasets by searching for specific questions, topics, and geographical coverage. Disco is intended to provide means to describe microdata by essential metadata for the discovery purpose.
 The DDI (Data Documentation Initiative) is a structured metadata standard related to the observation and measurement of human activity.

For a publication of DDI metadata as Linked Data, widely accepted and adopted RDF vocabularies (e.g. RDF Data Cube, DCAT, SKOS, and Physical Data Description) are reused to a large extend. 
The workshop organizers show how Disco is interwoven with these vocabularies. 
The combined usage of Physical Data Description, Disco, and DCAT enables the creation of data repositories which provide metadata for the description of collections, for data discovery, and for processing of the data.
 
In different real world use cases, we demonstrate the adoption and inclusion of Disco in existing information systems. 
Existing DDI instances are stored in form of XML documents which can be transformed into DDI-RDF and therefore exposed as Linked Data.
We describe typical constraints on metadata of DDI metadata, as well as how to formulate them by different constraint languages (e.g. OWL 2) and how to actually validate them.

\keywords{DDI-RDF Discovery Vocabulary, Disco, Microdata, Microdata Metadata, Discovery of Microdata, Data Documentation Initiative, DDI, Linked Data, Semantic Web}
\end{abstract}
%

\section{Anweisungen}

Proposals should not exceed 5 pages using a reasonable font size (not less than 11 pts). They should include the call for participation together with the following details:
- Abstract: 200 words maximum, for inclusion to the ESWC 2015 Web site.
- Tutorial description: objectives of the tutorial and relevance to ESWC 2015; information about the scope and level of detail of the material to be covered; intended audiences; learning objectives; practical sessions.
- Tutorial length: Half or full day, including preferences.
- Previous versions or related tutorials: other venues of this or similar tutorials (potentially by a different team of tutors), motivation for offering the tutorial again, at the ESWC2015.
- Tutoring team: short bios of the presenters, including previous training and public speaking experience (e.g., teaching in English, conference presentations, tutorials etc.)

\section{Call for Participation}

The DDI-RDF Discovery Vocabulary (Disco) is an RDF Schema vocabulary that supports the discovery of microdata sets and related metadata using RDF technologies in the Web of Linked Data. Disco can be used to discover datasets by searching for specific questions, topics, and geographical coverage. Disco is intended to provide means to describe microdata by essential metadata for the discovery purpose.
 
The DDI (Data Documentation Initiative) is a structured metadata standard related to the observation and measurement of human activity.
It started out in the mid-1990s as a replacement for traditional archival code books documenting research data, and then branched off to cover the research data life cycle. Over time, as the data landscape has
changed, the DDI XML specifications have evolved to add new coverage and functionality to respond to new user requirements.
While the Data Documentation Initiative (DDI) is an international metadata standard with origins in the quantitative social sciences, it is increasingly being used by researchers and practitioners in other disciplines. The DDI specifications are also being used to document other data types, such as social media, biomarkers, administrative data, and transaction data. The specification itself is modular and can document and manage different stages of the data lifecycle, such as conceptualization, collection, processing, analysis, distribution, discovery, repurposing, and archiving.
Existing DDI-XML instances can be transformed into this RDF format and therefore exposed as Linked Data. This workshop aims to present the Disco vocabulary and its practical appliance with DDI metadata in detail.
 
For a publication of DDI metadata as Linked Data, widely accepted and adopted RDF vocabularies (e.g. RDF Data Cube, DCAT, and PHDD) are reused to a large extend. The workshop organizers show how Disco is interwoven with these vocabularies. The Data Cube vocabulary is a W3C standard for representing data cubes representing multidimensional aggregate data derived from microdata which is represented by Disco. DCAT is a W3C standard for describing catalogs of datasets. Physical data description (PHDD) represents data (tables) in a rectangular format. The data could be either represented in records with character-separated values (CSV) or in records with fixed length. The combined usage of PHDD, Disco, and DCAT enables the creation of data repositories which provide metadata for the description of collections, for data discovery, and for processing of the data.
 
In different real world use cases, the workshop organizers demonstrate the adoption and inclusion of Disco in existing information systems. Workshop participants are encouraged to present their own use cases where Disco has been applied or will be used. Together with the workshop organizers, workshop participants will have the possibility to elaborate an RDF representation of their use cases and to formulate typical queries which are necessary to solve use case related problems.
The workshop has a technical perspective and addresses software developers as well as data publishers who want to publish DDI metadata as Linked Data. No prerequisites are necessary, since a short introduction to RDF and Linked Open Data will be given.

Existing DDI instances are stored in form of XML documents and validated against XML Schemas\footnote{XML Schemas of current DDI version available at \url{http://www.ddialliance.org/Specification/DDI-Lifecycle/3.2/XMLSchema/}}.
DDI-XML documents can be transformed into DDI-RDF and therefore exposed as Linked Data.
For XML, XML Schema is the standard constraint language to validate XML documents.
For RDF, however, there are multiple candidates for languages to express constraints on RDF data - but there is no standard language for RDF validation.
In 2014, two working groups on RDF validation have been established: the W3C RDF Data Shapes working group\footnote{\url{http://www.w3.org/2014/rds/charter}} and the DCMI RDF Application Profiles working group\footnote{\url{http://wiki.dublincore.org/index.php/RDF-Application-Profiles}}.
These working groups address the formulation and validation of RDF constraints.
We will describe typical constraints on metadata of DDI metadata, as well as how to formulate and to validate them.
We will show how to formulate these constraints by different constraint languages such as OWL 2.

\section{Tutorial Description}

\subsection{Objectives of the Tutorial} 

\subsection{Relevance to ESWC 2015}
In this tutorial, the DDI-RDF Discovery vocabulary (Disco) and its interplay with other existing vocabularies is introduced. In contrast to other newly developed vocabularies that can be adopted easily by Linked Data experts in most cases, the complexity of Disco is higher because of its domain-specific origin and the tight interplay with other existing vocabularies. This makes Disco more difficult to be adopted by Linked Data experts that may not be familiar with the domain-specific background of person-level micro data. However, since the number of organizations that aim to publish their complex data as Linked Data grows and the number of researchers that want to conduct their research with real-world data grows, a tutorial on representing such highly complex and inter-connected data sets may be a beneficial addition for the audience of ESWC 2015.

\subsection{Scope and Level of Detail of the Material}

\subsection{Intended Audiences}
Linked Data experts that are interested in modelling and publishing complex and highly inter-connected data sets as Linked Data

\subsection{Learning Objectives}
The learning objectives of this tutorial are the following:
\begin{itemize} 
\item Introduction of Disco for modelling complex person-level micro data
\item Representing the interplay of Disco data with existing vocabularies
\item Validation of Disco
\end{itemize}


\subsection{Practical Sessions}
During the tutorial, one complex data set will be built, which serves as ongoing running example. In all sessions, there will be practical exercises where the participants are asked to model and query particular parts of the example data sets. The infrastructure for these exercises, i.e. a triple store with example data, will be provided for all participants by the presenters.

\section{Tutorial Length}

We propose a full-day workshop. In the previous tutorial about disco, we experienced that a half-day tutorial offers only enough time to cover the most important aspects about disco and to include few hands-on exercises. Since we now want to expand the focus of the tutorial to the interplay of disco with other vocabularies and its validation, a full-day seems to be appropriate. Furthermore, we want to offer hands-on exercises again, because they were well received at the last tutorial.

\section{Previous Versions or Related Tutorials}

Thomas Bosch and Benjamin Zapilko held a half-day tutorial\footnote{tutorial slides avaliable at: \url{http://de.slideshare.net/boschthomas/201412-lets-disco-eddi-2014} and \url{http://de.slideshare.net/boschthomas/201412-lets-disco-2-eddi-2014}} at the EDDI14 – 6th Annual European DDI User Conference in London\footnote{\url{http://www.eddi-conferences.eu/ocs/index.php/eddi/eddi14/schedConf/program}}.
The tutorial was well received and the presenters got overall positive feedback. Since the audience of disco is two-fold - data professionals and the DDI community at the one hand and Linked Data experts at the other hand - the idea was to offer this tutorial at the ESWC conference as well.
At the EDDI conference, the first part of the audience - data professionals and the DDI community - was addressed.
At the ESWC conference, the second part of our audience - Linked Data experts - will be addressed.

\section{Tutoring Team}

%short bios of the presenters, including previous training and public speaking experience (e.g., teaching in English, conference presentations, tutorials etc.)

Thomas Bosch\footnote{\url{http://boschthomas.blogspot.com}} (M.Sc. in Computer Science) is a PhD student at Gesis - Leibniz Institute for the Social Sciences in Mannheim , Germany since 2010.
He studied Information Systems at the Technical University of Munich.
His general research topic is about RDF validation. 
He held multiple conference and workshop presentations (e.g. ISWC, DC, WWW, Dagstuhl seminars) in the last five years. 

Benjamin Zapilko holds a degree in Computer Sceince and has recently finished his PhD thesis about methods for matching social science relevant Linked Data. The defense talk will be at the end of January. Beside his PhD research, he works on modelling and publishing social science relevant data as Linked Data with the focus on data modelling and integration of heterogeneous data sources and information types. He presented his work at several international conferences.

\bibliography{../../literature/literature}{}
\bibliographystyle{plain}
\setcounter{tocdepth}{1}
%\listoftodos
\end{document}
