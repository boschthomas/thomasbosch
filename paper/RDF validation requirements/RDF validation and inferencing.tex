% This is LLNCS.DEM the demonstration file of
% the LaTeX macro package from Springer-Verlag
% for Lecture Notes in Computer Science,
% version 2.4 for LaTeX2e as of 16. April 2010
%
\documentclass{llncs}

% allows for temporary adjustment of side margins
\usepackage{chngpage}

% just makes the table prettier (see \toprule, \bottomrule, etc. commands below)
\usepackage{booktabs}

\usepackage[utf8]{inputenc}

% URL handling
\usepackage{url}
\urlstyle{same}

% Todos
%\usepackage[colorinlistoftodos]{todonotes}
%\newcommand{\ke}[1]{\todo[size=\small, color=orange!40]{\textbf{Kai:} #1}}
%\newcommand{\tb}[1]{\todo[size=\small, color=green!40]{\textbf{Thomas:} #1}}


%\usepackage{makeidx}  % allows for indexgeneration

%\usepackage{amsmath}
\usepackage{amsmath, amssymb}
\usepackage{mathabx}

% monospace within text
\newcommand{\ms}[1]{\texttt{#1}}

% examples
\usepackage{fancyvrb}
\DefineVerbatimEnvironment{ex}{Verbatim}{numbers=left,numbersep=2mm,frame=single,fontsize=\scriptsize}
\DefineVerbatimEnvironment{dl}{}{fontsize=\scriptsize}

\newenvironment{gcotable-old}{
  \scriptsize
  \sffamily
  \vspace{0.3cm}\linebreak
  \begin{tabular}{l|l|l|l|l|l}
  \hline
  \textbf{c. type} & \textbf{context concept} & \textbf{p. list} & \textbf{concepts} & \textbf{c. element} & \textbf{c. value} \\
  \hline

}{
  \hline
  \end{tabular}
  \linebreak
}

\newenvironment{gcotable}{
  \scriptsize
  \sffamily
  \vspace{0.3cm}
  \begin{tabular}{l|l|l|l|l|l|l}
  \hline
  \textbf{c. type} & \textbf{context concept} & \textbf{left p. list} & \textbf{right p. list} & \textbf{concepts} & \textbf{c. element} & \textbf{c. value} \\
  \hline

}{
  \hline
  \end{tabular}
  \linebreak
}

\newenvironment{DL}{
  \sffamily
  \vspace{0.3cm}
  \begin{tabular}{l}

}{
  \end{tabular}
  \linebreak
}

\newenvironment{evaluation}{
  \scriptsize
  \sffamily
  \vspace{0.3cm}
  \begin{tabular}{l|c|c|c|c|c}
  \hline
  \textbf{constraint} & \textbf{OWL2-DL} & \textbf{OWL2-QL} & \textbf{ReSh} & \textbf{ShEx} & \textbf{SPIN} \\
  \hline

}{
  \hline
  \end{tabular}
  \linebreak
}

\usepackage{xspace}
% Einfache und doppelte Anfuehrungszeichen
\newcommand{\qs}{``} 
\newcommand{\qe}{''\xspace} 
\newcommand{\sqs}{`} 
\newcommand{\sqe}{'\xspace} 

% checkmark
\usepackage{tikz}
\def\checkmark{\tikz\fill[scale=0.4](0,.35) -- (.25,0) -- (1,.7) -- (.25,.15) -- cycle;} 

% Xs
\usepackage{pifont}

% Tabellenabstände kleiner
\setlength{\intextsep}{10pt} % Vertical space above & below [h] floats
\setlength{\textfloatsep}{10pt} % Vertical space below (above) [t] ([b]) floats
% \setlength{\abovecaptionskip}{0pt}
% \setlength{\belowcaptionskip}{0pt}

\usepackage{tabularx}
\newcommand{\hr}{\hline\noalign{\smallskip}} % für die horizontalen linien in tabellen

% pipe
%\usepackage[T1]{fontenc}

% Todos
\usepackage[colorinlistoftodos]{todonotes}
\newcommand{\tb}[1]{\todo[size=\small, color=blue!40]{\textbf{Thomas:} #1}}
\newcommand{\ke}[1]{\todo[size=\small, color=orange!40]{\textbf{Kai:} #1}}
\newcommand{\an}[1]{\todo[size=\small, color=green!40]{\textbf{Andreas:} #1}}
\newcommand{\er}[1]{\todo[size=\small, color=red!40]{\textbf{Erman:} #1}}

\setcounter{secnumdepth}{5}

\begin{document}

%
%
\title{XXXXX}
%
\titlerunning{XXXXX}  % abbreviated title (for running head)
%                                     also used for the TOC unless
%                                     \toctitle is used
%
\author{XXXXX\inst{1} \and XXXXX\inst{2}}
%
\authorrunning{XXXXX} % abbreviated author list (for running head)
%
%%%% list of authors for the TOC (use if author list has to be modified)
\institute{XXXXX\\
\email{XXXXX},\\ 
\and
XXXXX \\
\email{XXXXX} 
}

\maketitle              % typeset the title of the contribution

\begin{abstract}


\keywords{..}
\end{abstract}
%

% ---------------

\section{Property Domain}

The property domain constraint

\begin{DL}
$\exists$ :studentOf . $\top$ $\sqsubseteq$ :JediStudent 
\end{DL}

restricts that individuals having :studentOf relationships must be Jedi students.
This property domain constraint can also be expressed by OWL 2 QL:

\begin{ex}
:studentOf rdfs:domain :JediStudent .
\end{ex}

Without reasoning, the data \ms{:Anakin :studentOf :Obi-Wan} is invalid and causes a constraint violation, as it is not explicitly stated that \ms{:Anakin} is assigned to the class \ms{:Jedi}. 
When inferencing is performed before validating, the class assignment \ms{:Anakin rdf:type :Jedi} is inferred which prevents the constraint violation to be raised.

\tb{complexity with OWL 2 QL inferencing?}

\tb{complexity without OWL 2 QL inferencing?}







\bibliography{../../literature/literature}{}
\bibliographystyle{plain}
\setcounter{tocdepth}{1}
%\listoftodos
\end{document}
