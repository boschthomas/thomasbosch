% This is LLNCS.DEM the demonstration file of
% the LaTeX macro package from Springer-Verlag
% for Lecture Notes in Computer Science,
% version 2.4 for LaTeX2e as of 16. April 2010
%
\documentclass{llncs}

% allows for temporary adjustment of side margins
\usepackage{chngpage}

% just makes the table prettier (see \toprule, \bottomrule, etc. commands below)
\usepackage{booktabs}

\usepackage[utf8]{inputenc}
%\usepackage[font=small,skip=0pt]{caption}

% footnotes
\usepackage{scrextend}

% URL handling
\usepackage{url}
\urlstyle{same}

% Todos
%\usepackage[colorinlistoftodos]{todonotes}
%\newcommand{\ke}[1]{\todo[size=\small, color=orange!40]{\textbf{Kai:} #1}}
%\newcommand{\tb}[1]{\todo[size=\small, color=green!40]{\textbf{Thomas:} #1}}


%\usepackage{makeidx}  % allows for indexgeneration

%\usepackage{amsmath}
\usepackage{amsmath, amssymb}
\usepackage{mathabx}

% monospace within text
\newcommand{\ms}[1]{\texttt{#1}}

% examples
\usepackage{fancyvrb}
\DefineVerbatimEnvironment{ex}{Verbatim}{numbers=left,numbersep=2mm,frame=single,fontsize=\scriptsize}

\usepackage{xspace}
% Einfache und doppelte Anfuehrungszeichen
\newcommand{\qs}{``} 
\newcommand{\qe}{''\xspace} 
\newcommand{\sqs}{`} 
\newcommand{\sqe}{'\xspace} 

% checkmark
\usepackage{tikz}
\def\checkmark{\tikz\fill[scale=0.4](0,.35) -- (.25,0) -- (1,.7) -- (.25,.15) -- cycle;} 

% Xs
\usepackage{pifont}

% Tabellenabstände kleiner
\setlength{\intextsep}{10pt} % Vertical space above & below [h] floats
\setlength{\textfloatsep}{10pt} % Vertical space below (above) [t] ([b]) floats
% \setlength{\abovecaptionskip}{0pt}
% \setlength{\belowcaptionskip}{0pt}

\usepackage{tabularx}
\newcommand{\hr}{\hline\noalign{\smallskip}} % für die horizontalen linien in tabellen

% Todos
\usepackage[colorinlistoftodos]{todonotes}
\newcommand{\bz}[1]{\todo[size=\small, color=orange!40]{\textbf{Ben:} #1}}
\newcommand{\tb}[1]{\todo[size=\small, color=green!40]{\textbf{Thomas:} #1}}

\newenvironment{table-1cols}{
  \scriptsize
  \sffamily
  \vspace{0.3cm}
  \begin{tabular}{l}
  \hline
  \textbf{Requirements} \\
  \hline

}{
  \hline
  \end{tabular}
  \linebreak
}

\newenvironment{table-2cols}{
  \scriptsize
  \sffamily
  \vspace{0.3cm}
  \begin{tabular}{l|l}
  \hline
  \textbf{Requirements} & \textbf{Covering DSCLs} \\
  \hline

}{
  \hline
  \end{tabular}
  \linebreak
}

\newenvironment{complexity}{
  %\scriptsize
  %\sffamily
  %\vspace{0.3cm}
  \begin{tabular}{l|l}
  \hline
  \textbf{Complexity Class} & \textbf{Complexity} \\
  \hline

}{
  \hline
  \end{tabular}
  \linebreak
}

\newenvironment{DL}{
  %\scriptsize
  %\sffamily
  \vspace{0cm}
  \begin{tabular}{r l}

}{
  \end{tabular}
  %\linebreak
}


\newenvironment{evaluation}{
  %\scriptsize
  %\sffamily
  %\vspace{0.3cm}
  \begin{tabular}{l|c|c|c|c|c|c}
  \hline
  \textbf{Constraint Class} & \textbf{DSP} & \textbf{OWL2-DL} & \textbf{OWL2-QL} & \textbf{ReSh} & \textbf{ShEx} & \textbf{SPIN} \\
  \hline

}{
  \hline
  \end{tabular}
  \linebreak
}

\newenvironment{constraint-languages-complexity}{
  %\scriptsize
  %\sffamily
  %\vspace{0.3cm}
  \begin{tabular}{l|c|c|c|c|c|c}
  \hline
  \textbf{Complexity Class} & \textbf{DSP} & \textbf{OWL2-DL} & \textbf{OWL2-QL} & \textbf{ReSh} & \textbf{ShEx} & \textbf{SPIN} \\
  \hline

}{
  \hline
  \end{tabular}
  \linebreak
}

\newenvironment{user-fiendliness}{
  %\scriptsize
  %\sffamily
  %\vspace{0.3cm}
  \begin{tabular}{l|c|c|c|c|c}
  \hline
  \textbf{criterion} & \textbf{DSP} & \textbf{OWL2} & \textbf{ReSh} & \textbf{ShEx} & \textbf{SPIN} \\
  \hline

}{
  \hline
  \end{tabular}
  \linebreak
}

\setcounter{secnumdepth}{5}

\begin{document}

%
%
\title{Let’s Disco – Publishing person-level data as Linked Data}
\subtitle{Building domain-specific Data Sets with Disco, PHDD, XKOS and widely used Vocabularies
}
%
\titlerunning{XXXXX}  % abbreviated title (for running head)
%                                     also used for the TOC unless
%                                     \toctitle is used
%
\author{Benjamin Zapilko\inst{1} \and Thomas Bosch\inst{1} \and Joachim Wackerow\inst{1}}
%
\authorrunning{XXXXX} % abbreviated author list (for running head)
%
%%%% list of authors for the TOC (use if author list has to be modified)
\institute{GESIS – Leibniz Institute for the Social Sciences, Germany\\
\email{firstname.lastname}@gesis.org}

\maketitle              % typeset the title of the contribution

\begin{abstract}
The DDI-RDF Discovery Vocabulary (Disco) is an RDF Schema vocabulary that enables the representation of person-level data sets and related metadata.
It is based on DDI (Data Documentation Initiative) which is a metadata standard related to the observation and measurement of human activity.
Using Disco, such data sets which are often used in different domains like the social, economic and behavioral sciences can be published as Linked Data. 
However, for a meaningful and complete publication of these data sets, additional established RDF vocabularies (e.g. RDF Data Cube, DCAT, SKOS, PROV-O, XKOS, and PHDD) are reused to a large extent. 
This combination allows for expressing information on metadata of data collections, single data sets, relationships between them, different versions of them and related provenance information. 
In this tutorial, we show how Disco is interwoven with these vocabularies and how these connections can be used in order to publish complex and domain-specific data sets.
In different real world use cases, we demonstrate the adoption and inclusion of Disco in existing information systems. 
%Existing DDI instances are stored in form of XML documents which can be transformed into DDI-RDF and therefore exposed as Linked Data.
%Additionally, we present typical metadata constraints of DDI metadata as well as how to formulate them by different constraint languages (e.g. OWL 2) and how to actually validate them.

 

\keywords{DDI-RDF Discovery Vocabulary, Disco, Person-level Metadata, Data Documentation Initiative, Linked Data}
\end{abstract}
%

%\section{Anweisungen}
%
%Proposals should not exceed 5 pages using a reasonable font size (not less than 11 pts). They should include the call for participation together with the following details:
%- Abstract: 200 words maximum, for inclusion to the ISWC 2015 Web site.
%- Tutorial description: objectives of the tutorial and relevance to ISWC 2015; information about the scope and level of detail of the material to be covered; intended audiences; learning objectives; practical sessions.
%- Tutorial length: Half or full day, including preferences.
%- Previous versions or related tutorials: other venues of this or similar tutorials (potentially by a different team of tutors), motivation for offering the tutorial again, at the ISWC2015.
%- Tutoring team: short bios of the presenters, including previous training and public speaking experience (e.g., teaching in English, conference presentations, tutorials etc.)

\section{Motivation}
The DDI-RDF Discovery Vocabulary (Disco) \cite{Bosch2012} is an RDF Schema vocabulary which can be utilized to represent person-level data sets and related metadata as Linked Data. 
It is based on the metadata standard DDI (Data Documentation Initiative) which enables the representation of the observation and measurement of human activity and which is often used in domains like the social, economic and behavioral sciences. 
The metadata of such data can get highly complex by e.g. necessary provenance information and relationships to data aggregates, data collections, and the physical data sources.
Hence beside Disco, additional widely accepted and adopted RDF vocabularies (e.g. RDF Data Cube\footnote{\url{http://www.w3.org/TR/vocab-data-cube/}}, DCAT\footnote{\url{http://www.w3.org/TR/vocab-dcat/}}, SKOS\footnote{\url{http://www.w3.org/TR/skos-reference/}}, and PROV-O\footnote{\url{http://www.w3.org/TR/prov-o/}}) are reused to a large extent \cite{Bosch2013} in order to enable a full data documentation, which is necessary to represent such data in full extent. 
In this tutorial, an introduction to Disco and its inter-connections to these vocabularies is given. 
It is presented how these connections can be used in order to publish complex domain-specific data sets.
In different real world use cases and by building a data set serving as running example, the presenters of the tutorial demonstrate the adoption and inclusion of Disco in existing information systems.
%During practical exercises, participants will have the possibility to elaborate an RDF representation of the running example or their own data sets and to formulate typical queries.
Participants are encouraged to present their own data sets and use cases where Disco may be applied. 


\textbf{Relevance to ISWC 2015.}
%In this tutorial, the DDI-RDF Discovery vocabulary (Disco) and its interplay with other existing vocabularies is introduced.
In contrast to other newly developed vocabularies that can be adopted easily in most cases, the complexity of Disco is higher because of its domain-specific origin and the tight interplay with other existing vocabularies. This makes Disco more difficult to be adopted by developers who may not be familiar with the domain-specific background of person-level or similar data. However, since the number of organizations that aim to publish their data as Linked Data grows and the number of Linked Data researchers that want to use real-world data grows, a tutorial on representing such inter-connected domain-specific data sets may be a beneficial addition for the audience of ISWC 2015.
This tutorial addresses Linked Data developers (beginners and experts) who want to publish Linked Data sets based on person-level or similar data as well as researchers who want to use such data sets for their research.

\textbf{Previous Versions or Related Tutorials.}
Benjamin Zapilko and Thomas Bosch held a half-day tutorial at the EDDI14 – 6th Annual European DDI User Conference in London\footnote{\url{http://www.eddi-conferences.eu/ocs/index.php/eddi/eddi14}}.
The tutorial was well received and the presenters got overall positive feedback. Since the audience of Disco is two-fold - data professionals and the DDI community at the one hand and the Linked Data community at the other hand - the idea was to offer this tutorial at the ISWC conference as well in order to address the Linked Data community in particular.
%At the EDDI conference, the first part of the audience - data professionals and the DDI community - was addressed.
%At the ISWC conference, the second part of our audience - Linked Data experts - will be addressed.

\section{Detailed Description}

The DDI-RDF Discovery Vocabulary (Disco)\cite{Bosch2012,Bosch2013} is an RDF Schema vocabulary that enables the representation of person-level data sets and related metadata as Linked Data.
It is based on DDI (Data Documentation Initiative)\footnote{\url{http://www.ddialliance.org/}}, a structured metadata standard related to the observation and measurement of human activity.
%DDI started out in the mid-1990s as a replacement for traditional archival code books documenting research data, and then branched off to cover the research data life cycle. 
%Over time, as the data landscape has changed, the DDI XML specifications have evolved to add new coverage and functionality to respond to new user requirements.
While DDI is an international metadata standard with origins in the quantitative social sciences, it is increasingly being used by researchers and practitioners in other disciplines. It is also being used to document other data types, such as social media, biomarkers, administrative data, and transaction data. The specification itself is modular and can document and manage different stages of the data lifecycle, such as conceptualization, collection, processing, analysis, distribution, discovery, repurposing, and archiving.
This tutorial aims to present the Disco vocabulary, its relationship to other necessary RDF vocabularies, and its validation.
 
For a publication of domain-specific metadata as Linked Data, additional established RDF vocabularies (e.g. RDF Data Cube, DCAT, SKOS, PROV-O, XKOS, and PHDD) are reused to a large extent. It is shown how Disco is interwoven with these vocabularies. 
These connections enable the representation of various relevant aspects, e.g. aggregate data derived from person-level data (also known as micro data) by the RDF Data Cube vocabulary, PROV-O and Disco, data collections and catalogs (together with DCAT), and data (tables) in a rectangular format by using PHDD. XKOS can be used to represent statistical classifications.
%The Data Cube vocabulary is a W3C standard for representing data cubes representing multidimensional aggregate data derived from microdata which is represented by Disco. DCAT is a W3C standard for describing catalogs of datasets. Physical data description (PHDD) represents data (tables) in a rectangular format. The data could be either represented in records with character-separated values (CSV) or in records with fixed length. The combined usage of PHDD, Disco, and DCAT enables the creation of data repositories which provide metadata for the description of collections, for data discovery, and for processing of the data. 

%When publishing data sets, their validation is a necessary requirement.
%Existing DDI instances are stored in form of XML documents and validated against XML Schemas\footnote{XML Schemas of current DDI version available at \url{http://www.ddialliance.org/Specification/DDI-Lifecycle/3.2/XMLSchema/}}.
%DDI-XML documents can be transformed into DDI-RDF and therefore exposed as Linked Data.
%For XML, XML Schema is the standard constraint language to validate XML documents.
%For RDF, 
%While XML Schema is the standard constraint language to validate XML documents, there is no standard language for RDF validation.
%However, there are multiple candidates for languages to express constraints on RDF data.
%In 2014, two working groups on RDF validation have been established addressing these issues: the W3C RDF Data Shapes working group\footnote{\url{http://www.w3.org/2014/rds/charter}} and the DCMI RDF Application Profiles working group\footnote{\url{http://wiki.dublincore.org/index.php/RDF-Application-Profiles}}.
%These working groups address the formulation and validation of RDF constraints.
%We will describe typical metadata constraints of Disco data sets as well as how to formulate them using different constraint languages such as OWL 2 and to validate them.
%We will show how to formulate these constraints by different constraint languages such as OWL 2.

The development of Disco started in 2012 at a Dagstuhl seminar followed by additional seminars and workshops.  
In 2014, we finished the technical review of Disco after contacting multiple mailing lists from the Linked Data and the DDI communities. 
%The received feedback has been incorporated into the Disco specification.
We are planning to officially publish Disco after three years of development in the first half of the year 2015. 
%after an additional official technical review of the DDI Alliance\footnote{\url{http://www.ddialliance.org/}}.

\subsection{Objectives of the Tutorial and Learning Objectives.} 
The main objective of this tutorial is to introduce Disco and to provide assistence for creating Linked Data sets using Disco. Person-level or similar metadata is not restricted to a particular domain and may be an interesting data source for Linked Data developers and researchers.
Because of the complexity that comes along with such data, we especially want to demonstrate how Disco data sets can be connected with information represented using other existing vocabularies that are relevant in that context. Vice verse, we show how these other vocabualries and data represented with such can benefit from the interplay with Disco.
Another objective of the tutorial is to share our experience in developing a RDF vocabulary out of an existing and complex domain-specific data model like DDI. We provide insight into the development and engineering process for creating Disco.
Finally, since Disco is already being adopted in the DDI community, we aim to increase the use of Disco in the Linked Data community.
The learning objectives of this tutorial are the following: (1) introducing Disco, PHDD, and XKOS for modeling complex person-level data, and (2) representing the interplay of Disco data with existing and established vocabularies like RDF Data Cube, DCAT, and PROV-O.

\subsection{Tutorial Format and Presentation Style.}
%During the tutorial, we will build one real world data set which serves as ongoing running example. 
%The whole data set can be exported from the Microdata Information System\footnote{\url{http://www.gesis.org/missy/editor/}}. 
%In all sessions, there will be practical exercises where the participants are asked to model and query particular parts of this data set. 
%The infrastructure for these exercises, i.e. a triple store and the data set, will be provided for all participants.
The scope of the tutorial covers the introduction of the Disco vocabulary, its connected vocabularies PHDD and XKOS as well as their interplay with other established vocabularies.
%Additionally, the validation of Disco data sets will be covered.
Slides of a previous tutorial\footnote{tutorial slides avaliable at: \url{http://de.slideshare.net/boschthomas/201412-lets-disco-eddi-2014} and \url{http://de.slideshare.net/boschthomas/201412-lets-disco-2-eddi-2014}} 
will be reused and extended. We will also refer to the current Disco specification\footnote{current specification is available at \url{https://github.com/linked-statistics/disco-spec}}.
Material for the practical exercises will be provided by the presenters (see Section 2.6).
During the tutorial, one complex data set will be built, which serves as running example. There will be practical exercises where the participants are asked to model particular parts of the example data sets. The infrastructure for these exercises, i.e. a triple store with example data, will be provided for all participants by the presenters.

\section{Audience}
The intended audience of this tutorial is Linked Data developers - beginners as well as experts - who are interested in modeling and publishing complex domain-specific and highly inter-connected data sets as Linked Data. Additionally, we address researchers who want to use such data sets for, e.g., running reasoning or matching algorithms. Since the connection with other vocabularies like RDF Data Cube, DCAT and PROV-O plays a major role in this context, the tutorial is not restricted to participants who are interested in person-level or similar data. Experiences from our last tutorial have shown that the size of a school class (15-30 persons) seems to be a reasonable number of participants. However, depending on the number of participants we are able to adjust the proportion of the practical session in order to avoid losing too much time by conducting the exercises.

\section{Presenters}

%short bios of the presenters, including previous training and public speaking experience (e.g., teaching in English, conference presentations, tutorials etc.)

\textbf{Benjamin Zapilko} is a postdoctoral researcher and lead of the team Data Linking at the GESIS department Knowledge Technologies for the Social Sciences (WTS). He holds a degree in computer science and finished his doctoral research on publishing and matching of Linked Open Data in the application area of the social sciences at Mannheim University in early 2015. His research interests focus on the domain-specific application of semantic technologies and Linked Open Data. He presented his work at several international conferences and workshops. Benjamin Zapilko is member of the RDF Vocabularies Working Group of the DDI Alliance and co-organizer of the Library Track at the Ontology Alignment Evaluation Initiative (OAEI) since 2012. Together with Thomas Bosch, he held a tutorial about Disco at the EDDI14 – 6th Annual European DDI User Conference\footnote{\url{http://www.eddi-conferences.eu/ocs/index.php/eddi/eddi14}} in 2014.

\begin{itemize}
\item primary contact person 
 \item affiliation: GESIS - Leibniz Institute for the Social Sciences
  \item email address: benjamin.zapilko@gesis.org
	\item homepage: \url{http://www.gesis.org/}
\end{itemize}

\textbf{Thomas Bosch} studied information systems, holds a degree in Computer Science (M.Sc.), and is currently a PhD student in Computer Science.
His research topic is all about RDF validation.
He investigates how to validate any RDF constraints (formulated by arbitrary constraint languages) and how to express them generically. 
He presented his work at international conferences and workshops (e.g. WWW, ISWC, DC, and Dagstuhl seminars). 
Thomas Bosch is member of the RDF Vocabularies Working Group\footnote{\url{http://www.ddialliance.org/alliance/working-groups#RDF}} of the DDI Alliance,
part of the editorial board of the  DCMI RDF Application Profiles Task Group\footnote{\url{http://wiki.dublincore.org/index.php/RDF-Application-Profiles}}, 
editor of the specifications DDI-RDF Discovery Vocabulary (Disco)\footnote{\url{http://rdf-vocabulary.ddialliance.org/discovery.html}}
and Physical Data Description (PHDD)\footnote{A vocabulary to represent data in a rectangular format in RDF; \url{https://github.com/linked-statistics/physical-data-description}},
and contributes to the W3C RDF Data Shapes Working Group\footnote{\url{http://www.w3.org/2014/data-shapes/charter}}.

\begin{itemize}
  \item affiliation: GESIS - Leibniz Institute for the Social Sciences
  \item email address: thomas.bosch@gesis.org
	\item homepage: \url{http://boschthomas.blogspot.com/}; \url{http://www.gesis.org/}
\end{itemize}

\textbf{Joachim Wackerow} holds a degree in sociology. 
He is the vice chair of the DDI Alliance Technical Committee whose charge is to create, maintain, and enhance the DDI specification.
He is the chair of the DDI Alliance RDF Vocabularies Working Group whose charge is to develop RDF vocabularies for efficient use of DDI metadata in the context of the Semantic Web/Linked Data.
He organizes the annual European DDI user conferences and holds DDI introduction workshops.

\begin{itemize}
  \item affiliation: GESIS - Leibniz Institute for the Social Sciences
  \item email address: joachim.wackerow@gesis.org
	\item homepage: \url{http://www.gesis.org/}
\end{itemize}

%\section{\textcolor{red}{Tutorial Length}}

%We propose a full-day tutorial. In a previous tutorial about Disco, we experienced that a half-day tutorial offers only enough time to cover the most important aspects about Disco and to include few hands-on exercises. Since we now want to expand the focus of the tutorial to the interplay of Disco with other vocabularies and its validation, a full-day seems to be appropriate. Furthermore, we want to offer hands-on exercises again, because they were well received at the last tutorial.

\section{Requirements}

Besides a beamer and wifi access, there are no additional requirements for the tutorial. In case wifi access may not be available or interrupted, the intended hands on sessions will be presented on a locally installed triple store.

%\bibliography{../../literature/literature}{}
%\bibliographystyle{plain}
%\setcounter{tocdepth}{1}
\begin{thebibliography}{1}
\bibitem{Bosch2012}
Thomas Bosch, Richard Cyganiak, Joachim Wackerow and Benjamin Zapilko. Leveraging the {DDI} Model for Linked Statistical Data in the Social, Behavioural, and Economic Sciences. Proceedings of the 2012 International Conference on Dublin Core and 	Metadata Applications
\bibitem{Bosch2013}
Thomas Bosch, Benjamin Zapilko, Joachim Wackerow and Arofan Gregory. Towards the discovery of person-level data: reuse of vocabularies and related use cases. Proceedings of the International Workshop on Semantic Statistics (SemStats 2013) collocated with the 12th International Semantic Web Conference (ISWC-2013)
\end{thebibliography}
%\listoftodos
\end{document}
