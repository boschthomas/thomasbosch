% This is LLNCS.DEM the demonstration file of
% the LaTeX macro package from Springer-Verlag
% for Lecture Notes in Computer Science,
% version 2.4 for LaTeX2e as of 16. April 2010
%
\documentclass{llncs}

% allows for temporary adjustment of side margins
\usepackage{chngpage}

% just makes the table prettier (see \toprule, \bottomrule, etc. commands below)
\usepackage{booktabs}

\usepackage[utf8]{inputenc}
%\usepackage[font=small,skip=0pt]{caption}

% footnotes
\usepackage{scrextend}

% URL handling
\usepackage{url}
\urlstyle{same}

%\usepackage{makeidx}  % allows for indexgeneration

%\usepackage{amsmath}
\usepackage{amsmath, amssymb}
\usepackage{mathabx}
\usepackage{caption} 
\captionsetup[table]{skip=10pt}

% monospace within text
\newcommand{\ms}[1]{\texttt{#1}}

% examples
\usepackage{fancyvrb}
\DefineVerbatimEnvironment{ex}{Verbatim}{numbers=left,numbersep=2mm,frame=single,fontsize=\scriptsize}

\usepackage{xspace}
% Einfache und doppelte Anfuehrungszeichen
\newcommand{\qs}{``} 
\newcommand{\qe}{''\xspace} 
\newcommand{\sqs}{`} 
\newcommand{\sqe}{'\xspace} 

% checkmark
\usepackage{tikz}
\def\checkmark{\tikz\fill[scale=0.4](0,.35) -- (.25,0) -- (1,.7) -- (.25,.15) -- cycle;} 

% Xs
\usepackage{pifont}

% Tabellenabstände kleiner
\setlength{\intextsep}{10pt} % Vertical space above & below [h] floats
\setlength{\textfloatsep}{10pt} % Vertical space below (above) [t] ([b]) floats
% \setlength{\abovecaptionskip}{0pt}
% \setlength{\belowcaptionskip}{0pt}

\usepackage{tabularx}
\newcommand{\hr}{\hline\noalign{\smallskip}} % für die horizontalen linien in tabellen

% Todos
\usepackage[colorinlistoftodos]{todonotes}
\newcommand{\ke}[1]{\todo[size=\small, color=orange!40]{\textbf{Kai:} #1}}
\newcommand{\tb}[1]{\todo[size=\small, color=green!40]{\textbf{Thomas:} #1}}
\newcommand{\er}[1]{\todo[size=\small, color=red!40]{\textbf{Erman:} #1}}
\newcommand{\an}[1]{\todo[size=\small, color=blue!40]{\textbf{Andy:} #1}}

\newenvironment{table-1cols}{
  \scriptsize
  \sffamily
  \vspace{0.3cm}
  \begin{tabular}{l}
  \hline
  \textbf{Requirements} \\
  \hline

}{
  \hline
  \end{tabular}
  \linebreak
}

\newenvironment{table-2cols}{
  \scriptsize
  \sffamily
  \vspace{0.3cm}
  \begin{tabular}{l|l}
  \hline
  \textbf{Requirements} & \textbf{Covering DSCLs} \\
  \hline

}{
  \hline
  \end{tabular}
  \linebreak
}

\newenvironment{complexity}{
  %\scriptsize
  %\sffamily
  %\vspace{0.3cm}
  \begin{tabular}{l|l}
  \hline
  \textbf{Complexity Class} & \textbf{Complexity} \\
  \hline

}{
  \hline
  \end{tabular}
  \linebreak
}

\newenvironment{DL}{
  %\scriptsize
  %\sffamily
  \vspace{0cm}
  \begin{tabular}{r l}

}{
  \end{tabular}
  %\linebreak
}


\newenvironment{evaluation}{
  %\scriptsize
  %\sffamily
  %\vspace{0.3cm}
  \begin{tabular}{l|c|c|c|c|c|c}
  \hline
  \textbf{Constraint Class} & \textbf{DSP} & \textbf{OWL2-DL} & \textbf{OWL2-QL} & \textbf{ReSh} & \textbf{ShEx} & \textbf{SPIN} \\
  \hline

}{
  \hline
  \end{tabular}
  \linebreak
}

\newenvironment{constraint-languages-complexity}{
  %\scriptsize
  %\sffamily
  %\vspace{0.3cm}
  \begin{tabular}{l|c|c|c|c|c|c}
  \hline
  \textbf{Complexity Class} & \textbf{DSP} & \textbf{OWL2-DL} & \textbf{OWL2-QL} & \textbf{ReSh} & \textbf{ShEx} & \textbf{SPIN} \\
  \hline

}{
  \hline
  \end{tabular}
  \linebreak
}

\newenvironment{user-fiendliness}{
  %\scriptsize
  %\sffamily
  %\vspace{0.3cm}
  \begin{tabular}{l|c|c|c|c|c}
  \hline
  \textbf{criterion} & \textbf{DSP} & \textbf{OWL2} & \textbf{ReSh} & \textbf{ShEx} & \textbf{SPIN} \\
  \hline

}{
  \hline
  \end{tabular}
  \linebreak
}

\setcounter{secnumdepth}{5}

\begin{document}
\renewcommand{\arraystretch}{1.3}
%
%
\title{RDF Validation of Metadata \\ on Highly-Complex Person-Level Data}
\subtitle{}

\titlerunning{XXXXX}  % abbreviated title (for running head)
%                                     also used for the TOC unless
%                                     \toctitle is used
%
\author{Thomas Bosch\inst{1} \and Benjamin Zapilko\inst{1} \and Joachim Wackerow\inst{1} \and Kai Eckert\inst{2}}
%
\authorrunning{} % abbreviated author list (for running head)
%
%%%% list of authors for the TOC (use if author list has to be modified)
\institute{GESIS – Leibniz Institute for the Social Sciences, Germany\\
\email{\{firstname.lastname\}@gesis.org},\\ 
\and
University of Mannheim, Germany \\
\email{kai@informatik.uni-mannheim.de} 
}

\maketitle              % typeset the title of the contribution

\begin{abstract}
...


\keywords{RDF Validation, RDF Constraints, DDI-RDF Discovery Vocabulary, Disco, RDF Data Cube Vocabulary, Linked Data, Semantic Web}
\end{abstract}

\section{Introduction}

Bosch and Eckert initiated a comprehensive database\footnote{Publicly available at \url{http://purl.org/net/rdf-validation}.} on RDF validation requirements to collect case studies, use cases, and requirements \cite{BoschEckert2014}. 
It is continuously updated and used to evaluate and to compare various existing solutions for RDF constraint formulation and validation. 
Requirements are classified to provide a high-level view on different solutions and to facilitate a better understanding of the problem domain.
Bosch et al. identified in total 74 requirements to formulate RDF constraints; each of them corresponding to a constraint type. 
They recently published a technical report\footnote{Available at: \url{http://arxiv.org/abs/1501.03933}} in which they explain each requirement (constraint type) in detail and give examples for each (represented by listed constraint languages) \cite{BoschNolleAcarEckert2015}.

\section{Validation of Metadata on Person-Level Data Sets}

We state at least one Disco constraint for the majority of the constraint types.
Where appropriate constraint types are related to complementary requirements of the RDF validation requirements database.
Constraint types marked with an asterisk can be used as OWL 2 axioms.
Thus, reasoners may be used to infer implicit triples resolving constraint violations caused when these constraints are validated.

\subsection{Data Model Consistency}

Is the data consistent with the intended semantics of the data model?
Such validation rules ensure the integrity of the data according to the data model.

\subsection{Subsumption*}

A \emph{subclass axiom}\footnote{{\em R-100-SUBSUMPTION}} ({\em concept inclusion} in DL) states that the class \emph{C1} is a subclass of the class \emph{C2} - \emph{C1} is more specific than \emph{C2}, 
i.e. each resources of the class \emph{C1} must also be part of the class extension of \emph{C2}.

\begin{itemize}
	\item \textbf{{\em DISCO-C-XX:}} 
All {\em disco:Universe}s must also be {\em skos:Concept}s:

\begin{DL}
Universe $\sqsubseteq$ Concept
\end{DL}
\end{itemize}

\subsection{Class Equivalence*}

{\em Class Equivalence}\footnote{{\em R-3-EQUIVALENT-CLASSES}} asserts that two concepts have the same instances.
While synonyms are an obvious example of equivalent concepts, in practice one more
often uses concept equivalence to give a name to complex expressions \cite{Kroetzsch2012}.
Concept equivalence is indeed subsumption from left and right ($A \sqsubseteq B$ and $B \sqsubseteq A$ implies $A \equiv B$).

\begin{itemize}
	\item \textbf{{\em DISCO-C-XX:}}
All {\em sio:SIO\_000367} resources must also be {\em disco:Variable}s:

\begin{DL}
Variable $\equiv$ SIO\_000367
\end{DL}

The Semanticscience Integrated Ontology (SIO)\footnote{https://code.google.com/p/semanticscience/wiki/SIO} provides a simple, integrated ontology of types and relations for rich description of objects, processes and their attributes.
{\em sio:SIO\_000367} is a variable defined as a value that may change within the scope of a given problem or set of operations.
Thus, {\em sio:SIO\_000367} is equivalent to {\em disco:Variable}.
\end{itemize}

\subsection{Sub Properties*}

{\em Sub Properties}\footnote{\emph{R-54-SUB-OBJECT-PROPERTIES}, \emph{R-54-SUB-DATA-PROPERTIES}} state that the property \emph{P1} is a sub property of the property \emph{P2} - that is, if an individual \emph{x} is connected by \emph{P1} to an individual or a literal \emph{y}, then \emph{x} is also connected by \emph{P2} to \emph{y}. 

\begin{itemize}
	\item \textbf{{\em DISCO-C-XX:}}
If an individual \emph{x} is connected by {\em disco:fundedBy} to an individual \emph{y}, then \emph{x} is also connected by {\em dcterms:contributor} to \emph{y}. 

\begin{DL}
fundedBy $\sqsubseteq$ contributor 
\end{DL}
\end{itemize}

\subsection{Property Domains*}

{\em Property Domains}\footnote{{\em R-25-OBJECT-PROPERTY-DOMAIN}, {\em R-26-DATA-PROPERTY-DOMAIN}} ({\em domain restrictions on roles} in DL) restrict the domain of object and data properties.
The purpose is to declare that a given property is associated with a class. 
In OO terms this is the declaration of a member, field, attribute or association. 
$\exists R. \top \sqsubseteq C$ is the object property restriction where $R$ is the object property (role) whose domain is restricted to concept $C$.
{\em Property Domain} constraints are defined for each \emph{Disco} object and data property.

\begin{itemize}
	\item \textbf{{\em DISCO-C-XX:}} 
Only {\em disco:Question}s can have {\em disco:responseDomain} relationships.

\begin{DL}
$\exists$ responseDomain.$\top$ $\sqsubseteq$ Question 
\end{DL}
\end{itemize}

\subsection{Property Ranges*}

{\em Property Ranges}\footnote{{\em R-28-OBJECT-PROPERTY-RANGE}, {\em R-35-DATA-PROPERTY-RANGE}} ({\em range restrictions on roles} in DL) restrict the range of object and data properties.
$\top \sqsubseteq \forall R . C$ is the range restriction to the object property $R$ (restricted by the concept $C$). 

\begin{itemize}
	\item \textbf{{\em DISCO-C-XX:}} 
{\em disco:caseQuantity} relationships can only point to literals of the datatype {\em xsd:nonNegativeInteger}.

\begin{DL}
$\top$ $\sqsubseteq$ $\forall$ caseQuantity.nonNegativeInteger \\
\end{DL}
\end{itemize}

\subsection{Inverse Object Properties*}

\subsection{Symmetric Object Properties*}

\subsection{Asymmetric Object Properties*}

A property is asymmetric\footnote{{\em R-62-ASYMMETRIC-OBJECT-PROPERTIES}} if it is disjoint from its own inverse \cite{Kroetzsch2012}.
An object property asymmetry axiom states that the object property \emph{OP} is asymmetric - that is, if an individual \emph{x} is connected by \emph{OP} to an individual \emph{y}, then \emph{y} cannot be connected by \emph{OP} to \emph{x}. 

\begin{itemize}
	\item \textbf{{\em DISCO-C-XX:}} 
A {\em disco:Variable} may be based on a {\em disco:RepresentedVariable}.
A {\em disco:RepresentedVariable}, however, cannot be based on a {\em disco:Variable}.
This is a kind of mistake which may occur as a semantically equivalent object property for the other direction may also be possible ({\em disco:basisOf}).

\begin{DL}
$basedOn \sqcap basedOn^{-} \sqsubseteq \bot$ 
\end{DL}
\end{itemize}

\subsection{Reflexive Object Properties*}

\emph{Reflexive Object Properties}\footnote{\emph{R-59-REFLEXIVE-OBJECT-PROPERTIES}} (\emph{reflexive roles}, \emph{global reflexivity} in DL) can be expressed by imposing local reflexivity on the top concept \cite{Kroetzsch2012}.

\subsection{Irreflexive Object Properties*}

An object property is irreflexive\footnote{\emph{R-60-IRREFLEXIVE-OBJECT-PROPERTIES}} (\emph{irreflexive role} in DL) if it is never locally reflexive \cite{Kroetzsch2012}.
An object property irreflexivity axiom \emph{IrreflexiveObjectProperty( OPE )} states that the object property expression \emph{OPE} is irreflexive - that is, no individual is connected by \emph{OPE} to itself. 
In \emph{Disco}, every object property is irreflexive.

\begin{itemize}
  \item \textbf{{\em DISCO-C-XX:}}
Within the Disco context, no individual is connected by the object property {\em instrument} to itself

\begin{DL}
$\top$ $\sqsubseteq$ $\neg$ $\exists  instrument . Self$ 
\end{DL}
\end{itemize}

\subsection{Class-Specific Irreflexive Object Properties*}

A property is \emph{irreflexive} if it is never locally reflexive \cite{Kroetzsch2012}.
An object property irreflexivity axiom states that the object property \emph{OP} is irreflexive - that is, no individual is connected by \emph{OP} to itself.
\emph{Class-Specific Irreflexive Object Properties} are object properties which are irreflexive within a given context, e.g. a class. 

\begin{itemize}
  \item \textbf{{\em DISCO-C-XX:}}
Within the Disco context, {\em skos:Concept}s cannot be related via the object property {\em skos:boader} to themselves

\begin{DL}
Concept $\sqsubseteq$ $\neg$$\exists$ broader.Self. 
\end{DL}

	\item \textbf{{\em DISCO-C-XX:}}
Within the Disco context, {\em skos:Concept}s cannot be related via the object property {\em skos:narrower} to themselves. 

\begin{DL}
Concept $\sqsubseteq$ $\neg$$\exists$ narrower.Self. 
\end{DL}
\end{itemize}

\subsection{Disjoint Properties}

A \emph{disjoint properties axiom}\footnote{\emph{R-9-DISJOINT-PROPERTIES}} states that all of the properties are pairwise disjoint; 
that is, no individual \emph{x} can be connected to an individual \emph{y} by these properties. 

\subsection{Disjoint Classes}

{\em Disjoint Classes}\footnote{{\em R-7-DISJOINT-CLASSES}} state that all of the classes are pairwise disjoint; 
that is, no individual can be at the same time an instance of these disjoint classes.

\begin{itemize}
	\item \textbf{{\em DISCO-C-XX:}} 
All \emph{Disco} classes are defined to be pairwise disjoint.
The following DL statements holds for each pair of Disco classes:

\begin{DL}
Study $\sqcap$ Variable $\sqsubseteq$ $\perp$\\
\end{DL}
\end{itemize}

\subsection{Context-Specific Property Groups}

The \emph{Context-Specific Property Groups}\footnote{\emph{R-66-PROPERTY-GROUPS}} constraint groups data and object properties within a context (e.g. a class).

\subsection{Context-Specific Inclusive OR of Properties}

\subsection{Context-Specific Inclusive OR of Property Groups}

\subsection{Recursive Queries}

\subsection{Individual Inequality}

\subsection{Equivalent Properties*}

\subsection{Property Assertions}

\subsection{Data Property Facets}

For datatype properties it should be possible to declare frequently needed \emph{facets}\footnote{\emph{R-46-CONSTRAINING-FACETS}} to drive user interfaces and validate input against simple conditions, including min/max value, regular expressions, string length - similar to XSD datatypes. 
Constraining facets, to restrict datatypes of RDF literals, may be: \emph{xsd:length}, \emph{xsd:minLength}, \emph{xsd:maxLength}, \emph{xsd:pattern}, \emph{xsd:enumeration}, \emph{xsd:whiteSpace}, \\ \emph{xsd:maxInclusive}, \emph{xsd:maxExclusive}, \emph{xsd:minExclusive}, \emph{xsd:minInclusive}, \emph{xsd:total} \emph{Digits}, \emph{xsd:fractionDigits}.

\begin{itemize}
	\item \textbf{{\em DISCO-C-XX:}} The abstract of a study (\emph{disco:purpose}) should have a minimum length (\emph{xsd:minInclusive}) of x. 
\end{itemize}

\subsection{Literal Pattern Matching}

There are multiple use cases associated with the requirement to match literals according to given patterns\footnote{\emph{R-44-PATTERN-MATCHING-ON-RDF-LITERALS}}.

\begin{itemize}
	\item \textbf{{\em DISCO-C-XX:}} Each \emph{disco:Variable} of a given \emph{disco:LogicalDataSet} must have a given prefix for its variable name (\emph{skos:notation}). 
\end{itemize}

\subsection{Negative Literal Pattern Matching}

\begin{itemize}
	\item \textbf{{\em DISCO-C-XX:}} 
\end{itemize}

\subsection{Object Property Paths*}

\emph{Object Property Paths}\footnote{\emph{R-55-OBJECT-PROPERTY-PATHS}} (or \emph{Object Property Chains} and in DL terminology \emph{complex role inclusion axiom} or \emph{role composition}) is the more complex form of sub properties. 
This axiom states that, if an individual \emph{x} is connected by a sequence of object property expressions \emph{OPE$_1$, ..., OPE$_n$} with an individual \emph{y}, then \emph{x} is also connected with \emph{y} by the object property expression \emph{OPE}.  
Role composition can only appear on the left-hand side of complex role inclusions \cite{Kroetzsch2012}.

\subsection{Intersection*}

\subsection{Disjunction*}

A \emph{union class expression}\footnote{{\em R-17-DISJUNCTION-OF-CLASS-EXPRESSIONS} and {\em R-18-DISJUNCTION-OF-DATA-RANGES}} contains all individuals that are instances of at least one class $C_{i}$ for 1 $\leq$ i $\leq$ n. 
A \emph{union data range} contains all tuples of literals that are contained in at least one data range $DR_{i}$ for 1 $\leq$ i $\leq$ n.
Synonyms of {\em disjunction} are {\em union} and {\em inclusive or}.

\begin{itemize}
	\item \textbf{{\em DISCO-C-XX:}} 
Only {\em disco:Variable}s or {\em disco:Question}s or {\em disco:RepresentedVariable}s can have {\em disco:concept} relationships to {\em skos:Concept}s.

\begin{DL}
Variable $\sqcup$ Question $\sqcup$ RepresentedVariable $\sqsubseteq$ $\forall$ concept.Concept \\
\end{DL}
\end{itemize}






\subsection{Unqualified Cardinality Restrictions}

\subsection{Minimum Qualified Cardinality Restrictions}

A minimum cardinality restrictions contains all those individuals that are connected by a property to at least n different individuals/literals 
that are instances of a particular class or data range. If the class is missing, it is taken to be owl:Thing. 
If the data range is missing, it is taken to be rdfs:Literal.
$\geq n R. C$ is a minimum qualified cardinality restriction where $n \in \mathbb{N}$\footnote{{\em R-75-MINIMUM-QUALIFIED-CARDINALITY-ON-PROPERTIES} and {\em R-211-CARDINALITY-CONSTRAINTS}}.
\textbf{{\em DISCO-C-XX:}}
A {\em disco:Questionnaire} has at least 1 {\em disco:question} relationship to {\em disco:Question}\footnote{When mapped to DL, we do not state namespace prefixes for simplicity reasons}.

\begin{DL}
Questionnaire $\sqsubseteq$ $\geq$1 question.Question
\end{DL}

\subsection{Exact Qualified Cardinality Restrictions}

An exact cardinality restriction contains all those individuals that are connected by a property to exactly n different individuals that are instances of a particular class or data range. 
If the class is missing, it is taken to be owl:Thing. 
If the data range is not present, it is taken to be rdfs:Literal.
$\geq n R. C \sqcap \leq n R. C $ is an exact qualified cardinality restriction where $n \in \mathbb{N}$\footnote{{\em R-74-EXACT-QUALIFIED-CARDINALITY-ON-PROPERTIES} and {\em R-211-CARDINALITY-CONSTRAINTS}.}.
\textbf{{\em DISCO-C-XX:}}
A {\em disco:Question} has exactly 1 {\em disco:universe} relationship to {\em disco:Universe}.

\begin{DL}
Question $\sqsubseteq$ \\
$\geq$1 universe.Universe $\sqcap$ $\leq$1 universe.Universe \\
\end{DL}

\subsection{Universal Quantifications*}

A universal class expression ({\em value restriction} in DL) contains all those individuals that are connected by an object property only to individuals that are instances of a particular class\footnote{{\em R-91-UNIVERSAL-QUANTIFICATION-ON-PROPERTIES}}.
\textbf{{\em DISCO-C-XX:}}
Only {\em disco:LogicalDataSet}s can have {\em disco:aggregation} relationships to {\em qb:DataSet}s.

\begin{DL}
LogicalDataSet $\sqsubseteq$ $\forall$ aggregation.DataSet \\
\end{DL}

\subsection{Existential Quantifications}

An existential class expression ({\em existential restriction} in DL terminology) contains all those individuals that are connected by the property P to an individual x that is an instance of the class C or to literals that are in the data range DR\footnote{{\em R-86-EXISTENTIAL-QUANTIFICATION-ON-PROPERTIES}}.
\textbf{{\em DISCO-C-XX:}} 
There must be at least 1 {\em disco:universe} relationship from {\em disco:Studies} or {\em disco:StudyGroups} to {\em disco:Universe}.

\begin{DL}
Study $\sqcup$ StudyGroup $\sqsubseteq$ $\exists$ universe.Universe \\
\end{DL}

\subsection{Class-Specific Property Ranges}		

{\em Class-Specific Property Range} restricts the range of object and data properties for individuals within a specific context (e.g. class, shape, application profile).
The values of each member property of a class may be limited by their value type, such as xsd:string or foaf:Person\footnote{{\em R-29-CLASS-SPECIFIC-RANGE-OF-RDF-OBJECTS} and {\em R-36-CLASS-SPECIFIC-RANGE-OF-RDF-LITERALS}}. 
\textbf{{\em DISCO-C-XX:}} 
Only {\em disco:Question}s can have {\em disco:questionText} relationships to literals of the datatype {\em rdf:langString}.

\begin{DL}
$\neg$Question $\sqsubseteq$ $\neg\exists$ questionText.langString
\end{DL}

\subsection{Context-Specific Exclusive OR of Property Groups}

Exclusive or is a logical operation that outputs true whenever both inputs differ (one is true, the other is false).
Only one of multiple property groups leads to valid data\footnote{{\em R-13-DISJOINT-GROUP-OF-PROPERTIES-CLASS-SPECIFIC}}.
\textbf{{\em DISCO-C-XX:}}
Within the context of Disco, skos:Concepts can have either skos:definition (when interpreted as DDI concepts) or skos:notation and skos:prefLabel properties (when interpreted as DDI codes and categories), but not both.

\begin{DL}
Concept $\sqsubseteq$ ($\neg$ D $\sqcap$ C) $\sqcup$ (D $\sqcap$ $\neg$ C) \\ 
D $\equiv$ A $\sqcap$ B \\
A $\sqsubseteq$ $\geq$ 1 notation.string $\sqcap$ $\leq$ 1 notation.string \\
B $\sqsubseteq$ $\geq$ 1 prefLabel.string $\sqcap$ $\leq$ 1 prefLabel.string \\
C $\sqsubseteq$ $\geq$ 1 definition.string $\sqcap$ $\leq$ 1 definition.string \\
\end{DL}

\subsection{Allowed Values}

It is a common requirement to narrow down the value space of a property by an exhaustive enumeration of the valid values (both literals or resource). 
This is often rendered in drop down boxes or radio buttons in user interfaces. 
Allowed values for properties can be IRIs, IRIs (matching one or multiple patterns), (any) literals, literals of a list of allowed literals (e.g. 'red' 'blue' 'green'), typed literals one or multiple type(s) (e.g. xsd:string)\footnote{{\em R-30-ALLOWED-VALUES-FOR-RDF-OBJECTS} and 
{\em R-37-ALLOWED-VALUES-FOR-RDF-LITERALS}}.

\textbf{{\em DISCO-C-XX}}.
{\em disco:CategoryStatistics} can only have {\em disco:computationBase} relationships to the values 'valid' and 'invalid' of the datatype {\em rdf:langString}.

\begin{DL}
CategoryStatistics $\equiv$ \\ $\forall$ computationBase.\{valid,invalid\} $\sqcap$ langString \\
\end{DL}

\subsection{Membership in Controlled Vocabularies.}

Resources can only be members of listed controlled vocabularies\footnote{{\em R-32-MEMBERSHIP-OF-RDF-OBJECTS-IN-CONTROLLED-VOCABULARIES} and
{\em R-39-MEMBERSHIP-OF-RDF-LITERALS-IN-CONTROLLED-VOCABULARIES}}.

\textbf{{\em DISCO-C-XX}}.
{\em disco:SummaryStatistics} can only have {\em disco:summaryStatisticType} relationships to {\em skos:Concept}s which must be members of the controlled vocabulary {\em ddicv:SummaryStatisticType} which is a {\em skos:ConceptScheme}.

\begin{DL}
SummaryStatistics $\sqsubseteq$ \\
$\forall summaryStatisticType.Concept \sqcap \forall inScheme . A$ \\
$A \equiv ConceptScheme \sqcap \{SummaryStatisticType\}$
\end{DL}

\subsection{Literal Value Comparison}

Depending on the property semantics,
there are cases where two different literal values must have
a specific ordering with respect to an operator. 
P1 and P2 are the datatype properties we need to compare and 
OP is the comparison operator (\textless, \textless=, \textgreater, \textgreater=, =, !=)\footnote{{\em R-43-LITERAL-VALUE-COMPARISON}}.
The {\em COMP Pattern}, one of the Data Quality Test Pattern, can be used to validate the {\em Literal Value Comparison} constraint \cite{Kontokostas2014}:

\begin{ex}
SELECT ?s WHERE { 
    ?s %%P1%% ?v1 .
    ?s %%P2%% ?v2 .
    FILTER ( ?v1 %%OP%% ?v2 ) }
\end{ex}

\textbf{{\em DISCO-C-XX}}.
{\em disco:startDate}s must be before (‘\textless’) the {\em disco:endDate}s.
To validate this constraint we bind the variables as follows (P1: {\em disco:startDate}, P2: {\em disco:endDate}, OP: \textless). 

\subsection{Literal Ranges}

P1 is a data property (of an instance of class C1) and its literal value must be between the range of [$V_{min}$,$V_{max}$].
\textbf{{\em DISCO-C-XX:}}
{\em disco:percentage} (domain: {\em disco:CategoryStatistics}) literals must be of the datatype {\em xsd:double} whose range should be restricted to be between 0 and 100 (not expressible in DL)\footnote{{\em R-45-RANGES-OF-RDF-LITERAL-VALUES}}.

\subsection{Mathematical Operations}

Examples for {\em Mathematical Operations} are the addition of two dates, the addition of days to a start date, and statistical computations (e.g. average, mean, sum).
\textbf{{\em DISCO-C-XX:}}
The sum of {\em disco:percentage} {\em xsd:double} values of all codes (represented as {\em skos:Concept}s) of a code list ({\em skos:ConceptScheme} or {\em skos:OrderedCollection}), serving as representation of a particular {\em disco:Variable}, must exactly be 100  
(not expressible in DL)\footnote{{\em R-42-MATHEMATICAL-OPERATIONS} and {\em R-41-STATISTICAL-COMPUTATIONS}}.

\subsection{Default Values}

Default values for objects and literals are inferred automatically.
It should be possible to declare the default value for a given property, e.g. so that input forms can be pre-populated and to insert a required property that is missing in a web service call\footnote{{\em R-31-DEFAULT-VALUES-OF-RDF-OBJECTS} and {\em R-38-DEFAULT-VALUES-OF-RDF-LITERALS}}.
\textbf{{\em DISCO-C-XX:}}
The value 'true' for the property {\em disco:isPublic} ({\em xsd:boolean}) indicates that the data set ({\em disco:LogicalDataSet}) can be accessed (usually downloaded) by anyone.
Per default, access to data sets should be restricted ('false').

\subsection{Language Tag Matching}

For particular data properties, values must be stated for predefined languages (not expressible in DL)\footnote{{\em R-47-LANGUAGE-TAG-MATCHING}}.
\textbf{{\em DISCO-C-XX:}}
There must be an English variable name ({\em skos:notation}) for each {\em disco:Variable} within {\em disco:LogicalDataSet}s.

\subsection{Language Tag Cardinality}

For particular data properties, values of predefined languages must be stated for determined number of times (not expressible in DL)
\footnote{{\em R-49-RDF-LITERALS-HAVING-AT-MOST-ONE-LANGUAGE-TAG} and {\em R-48-MISSING-LANGUAGE-TAGS}}.
\textbf{{\em DISCO-C-XX:}}
There must be at least one English {\em disco:questionText} for each {\em disco:Question} within {\em disco:LogicalDataSet}s.
\textbf{{\em DISCO-C-XX:}}
There should be at most one English literal value for variable names ({\em skos:notation}, domain: {\em disco:Variable}).

\subsection{Conditional Properties}

If specific properties exist, then specific other properties also have to be present\footnote{{\em R-71-CONDITIONAL-PROPERTIES}}.
\textbf{{\em DISCO-C-XX:}}
If a {\em skos:Concept represents a code (having a {\em skos:notation} property) and a category (having a {\em skos:prefLabel} property), 
then the property {\em disco:isValid} has to be stated indicating if the code is valid ('true') or missing ('false').

\subsection{Recommended Properties}

Which properties are not required but recommended within a particular context\footnote{{\em R-72-RECOMMENDED-PROPERTIES}}.
\textbf{{\em DISCO-C-XX:}}
The property ({\em skos:notation} is not mandatory for {\em disco:Variable}s, but recommended to indicate variable names.

\subsection{Value is Valid for Datatype}

Make sure that a value is valid for its datatype.
It has to be ensured, e.g., that a date is really a date, or that a xsd:nonNegativeInteger value is not negative. 
\textbf{{\em DISCO-C-XX:}}
Check if literal values of the property {\em disco:startDate} are really of the datatype {\em xsd:date}(not expressible in DL).

\subsection{Functional Properties}

\subsection{Hierarchies}

Hierarchies of DDI concepts

\begin{itemize}
	\item root?
\end{itemize}

\subsection{Ordering}

In DDI, variables, questions, and categories are typically organized in a particular order. 
For obtaining this order, {\em skos:OrderedCollection} resources are used. 
A collection of variables, e.g, is represented as being of the type {\em skos:OrderedCollection} containing multiple variables (each represented as {\em skos:Concept}) in a {\em skos:memberList}. 

\section{Validation in Combination with other Vocabularies}

\subsection{RDF Data Cube Vocabulary}

There are 22 RDF Data Cube integrity constraints\footnote{http://www.w3.org/TR/vocab-data-cube/\#wf} \cite{CyganiakReynolds2014}.

\begin{itemize}
	\item \textcolor{blue}{describe 1 constraint of each type in detail with Datalog query}
\end{itemize}

\textbf{Data Model Consistency.}
{\em IC-0}, \textcolor{red}{{\em IC-8}}, {\em IC-13}, {\em IC-14}, {\em IC-15}, {\em IC-16}, {\em IC-17}, \textcolor{red}{{\em IC-18}}
%\begin{itemize}
	%\item {\em IC-0:} Datatypes must be consistent under RDF D-entailment using a datatype map containing all the datatypes used within the graph.
	%\item \textcolor{red}{{\em IC-8:} Every {\em qb:componentProperty} on a {\em qb:SliceKey} must also be declared as a {\em qb:component} of the associated {\em qb:DataStructureDefinition}.} 
%\end{itemize}

\textbf{Unqualified Cardinality Restrictions.}
{\em IC-4}, {\em IC-5}, {\em IC-10}, {\em IC-11}
%\begin{itemize}
	%\item {\em IC-4:} Every dimension declared in a {\em qb:DataStructureDefinition} must have a declared {\em rdfs:range}. 
	%\item {\em IC-5:} Every dimension with range {\em skos:Concept} must have a {\em qb:codeList}. 
	%\item {\em IC-10:} Every {\em qb:Slice} must have a value for every dimension declared in its {\em qb:sliceStructure}. 
	%\item {\em IC-11:} All dimensions required - Every {\em qb:Observation} has a value for each dimension declared in its associated {\em qb:DataStructureDefinition}.
%\end{itemize}

\textbf{Qualified Cardinality Restrictions.}
{\em IC-1}, {\em IC-2}, {\em IC-3}, {\em IC-6}, {\em IC-7}, {\em IC-9} 
%\begin{itemize}
	%\item {\em IC-1:} Every {\em qb:Observation} has exactly one associated {\em qb:DataSet}. 
	%\item {\em IC-2:} Every {\em qb:DataSet} has exactly one associated {\em qb:DataStructureDefinition}. 
	%\item {\em IC-3:} Every {\em qb:DataStructureDefinition} must include at least one declared measure. 
	%\item {\em IC-6:} The only components of a {\em qb:DataStructureDefinition} that may be marked as optional, using {\em qb:componentRequired} are attributes. 
	%\item {\em IC-7:} Every {\em qb:SliceKey} must be associated with a {\em qb:DataStructureDefinition}. 
	%\item {\em IC-9:} Each {\em qb:Slice} must have exactly one associated {\em qb:sliceStructure}. 
%\end{itemize}

\textbf{Duplicate Detection.}
{\em IC-12}
%\begin{itemize}
	%\item {\em IC-12:} No duplicate observations - No two {\em qb:Observations} in the same {\em qb:DataSet} may have the same value for all dimensions. 
%\end{itemize}

\textbf{Membership in Controlled Vocabularies.}
\textcolor{red}{{\em IC-19}}

\textbf{SKOS Hierarchies.}
{\em IC-20}, {\em IC-21}

%\textbf{Mandatory Properties.}

%\textbf{Optional Properties.}

\begin{itemize}
	\item \textcolor{blue}{are there further constraints which can be described in this paper?}
\end{itemize}

\subsection{SKOS and XKOS}

\subsection{PHDD}

\subsection{DCAT}

\section{Conclusion and Future Work}

\bibliography{../../literature/literature}{}
\bibliographystyle{plain}
\setcounter{tocdepth}{1}
%\listoftodos
\end{document}
