% This is LLNCS.DEM the demonstration file of
% the LaTeX macro package from Springer-Verlag
% for Lecture Notes in Computer Science,
% version 2.4 for LaTeX2e as of 16. April 2010
%
\documentclass{llncs}

% allows for temporary adjustment of side margins
\usepackage{chngpage}

% just makes the table prettier (see \toprule, \bottomrule, etc. commands below)
\usepackage{booktabs}

\usepackage[utf8]{inputenc}

% URL handling
\usepackage{url}
\urlstyle{same}

% Todos
%\usepackage[colorinlistoftodos]{todonotes}
%\newcommand{\ke}[1]{\todo[size=\small, color=orange!40]{\textbf{Kai:} #1}}
%\newcommand{\tb}[1]{\todo[size=\small, color=green!40]{\textbf{Thomas:} #1}}


%\usepackage{makeidx}  % allows for indexgeneration

%\usepackage{amsmath}
\usepackage{amsmath, amssymb}
\usepackage{mathabx}

% monospace within text
\newcommand{\ms}[1]{\texttt{#1}}

% examples
\usepackage{fancyvrb}
\DefineVerbatimEnvironment{ex}{Verbatim}{numbers=left,numbersep=2mm,frame=single,fontsize=\scriptsize}

\usepackage{xspace}
% Einfache und doppelte Anfuehrungszeichen
\newcommand{\qs}{``} 
\newcommand{\qe}{''\xspace} 
\newcommand{\sqs}{`} 
\newcommand{\sqe}{'\xspace} 

% checkmark
\usepackage{tikz}
\def\checkmark{\tikz\fill[scale=0.4](0,.35) -- (.25,0) -- (1,.7) -- (.25,.15) -- cycle;} 

% Xs
\usepackage{pifont}

% Tabellenabstände kleiner
\setlength{\intextsep}{10pt} % Vertical space above & below [h] floats
\setlength{\textfloatsep}{10pt} % Vertical space below (above) [t] ([b]) floats
% \setlength{\abovecaptionskip}{0pt}
% \setlength{\belowcaptionskip}{0pt}

\usepackage{tabularx}
\newcommand{\hr}{\hline\noalign{\smallskip}} % für die horizontalen linien in tabellen

% Todos
\usepackage[colorinlistoftodos]{todonotes}
\newcommand{\ke}[1]{\todo[size=\small, color=orange!40]{\textbf{Kai:} #1}}
\newcommand{\tb}[1]{\todo[size=\small, color=green!40]{\textbf{Thomas:} #1}}

\newenvironment{table-1cols}{
  \scriptsize
  \sffamily
  \vspace{0.3cm}
  \begin{tabular}{l}
  \hline
  \textbf{Requirements} \\
  \hline

}{
  \hline
  \end{tabular}
  \linebreak
}

\newenvironment{table-2cols}{
  \scriptsize
  \sffamily
  \vspace{0.3cm}
  \begin{tabular}{l|l}
  \hline
  \textbf{Requirements} & \textbf{Covering DSCLs} \\
  \hline

}{
  \hline
  \end{tabular}
  \linebreak
}

\newenvironment{DL}{
  \scriptsize
  \sffamily
  \vspace{0.3cm}
  \begin{tabular}{l}
	\textbf{DL:} \\

}{
  \end{tabular}
  \linebreak
}

\setcounter{secnumdepth}{5}

\begin{document}

%
%
\title{XXXXX}
%
\titlerunning{XXXXX}  % abbreviated title (for running head)
%                                     also used for the TOC unless
%                                     \toctitle is used
%
\author{XXXXX\inst{1} \and XXXXX\inst{2}}
%
\authorrunning{XXXXX} % abbreviated author list (for running head)
%
%%%% list of authors for the TOC (use if author list has to be modified)
\institute{XXXXX\\
\email{XXXXX},\\ 
\and
XXXXX \\
\email{XXXXX} 
}

\maketitle              % typeset the title of the contribution

\begin{abstract}


\keywords{..}
\end{abstract}
%

\section{Appendix}

\subsection{Allowed Usage of Constructs in Class Expressions in OWL 2 QL}

\textbf{Subclass Expressions}

\begin{ex}
subClassExpression :=
    Class |
    subObjectSomeValuesFrom | DataSomeValuesFrom
subObjectSomeValuesFrom := 'ObjectSomeValuesFrom' '(' ObjectPropertyExpression owl:Thing ')'
\end{ex}

\textbf{Superclass Expressions}

\begin{ex}
superClassExpression :=
    Class |
    superObjectIntersectionOf | superObjectComplementOf |
    superObjectSomeValuesFrom | DataSomeValuesFrom
\end{ex}
 
\subsection{Supported Constructs in OWL 2 QL}

\begin{itemize}
	\item subclass axioms (SubClassOf)
  \item class expression equivalence (EquivalentClasses)
  \item class expression disjointness (DisjointClasses)
  \item inverse object properties (InverseObjectProperties)
  \item property inclusion (SubObjectPropertyOf not involving property chains and SubDataPropertyOf)
  \item property equivalence (EquivalentObjectProperties and EquivalentDataProperties)
  \item property domain (ObjectPropertyDomain and DataPropertyDomain)
  \item property range (ObjectPropertyRange and DataPropertyRange)
  \item disjoint properties (DisjointObjectProperties and DisjointDataProperties)
  \item symmetric properties (SymmetricObjectProperty)
  \item reflexive properties (ReflexiveObjectProperty)
  \item irreflexive properties (IrreflexiveObjectProperty)
  \item asymmetric properties (AsymmetricObjectProperty)
  \item assertions: DifferentIndividuals, ClassAssertion, ObjectPropertyAssertion, and DataPropertyAssertion
\end{itemize}

\subsection{Not Supported Constructs in OWL 2 QL}

\begin{itemize}
	\item existential quantification to a class expression or a data range (ObjectSomeValuesFrom and DataSomeValuesFrom) in the subclass position
  \item self-restriction (ObjectHasSelf)
  \item existential quantification to an individual or a literal (ObjectHasValue, DataHasValue)
  \item enumeration of individuals and literals (ObjectOneOf, DataOneOf)
  \item universal quantification to a class expression or a data range (ObjectAllValuesFrom, DataAllValuesFrom)
  \item cardinality restrictions (ObjectMaxCardinality, ObjectMinCardinality, ObjectExactCardinality, DataMaxCardinality, DataMinCardinality, DataExactCardinality)
  \item disjunction (ObjectUnionOf, DisjointUnion, and DataUnionOf)
  \item property inclusions (SubObjectPropertyOf) involving property chains
  \item functional and inverse-functional properties (FunctionalObjectProperty, InverseFunctionalObjectProperty, and FunctionalDataProperty)
  \item transitive properties (TransitiveObjectProperty)
  \item keys (HasKey)
  \item individual equality assertions and negative property assertions (SameIndividual, NegativeObjectPropertyAssertion, NegativeDataPropertyAssertion)
\end{itemize}

\bibliography{../../literature/literature}{}
\bibliographystyle{plain}
\setcounter{tocdepth}{1}
%\listoftodos
\end{document}
